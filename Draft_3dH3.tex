%!TEX output_directory = aux

\documentclass[a4paper, 11pt]{article}
%%%%%%%%%%%%%%%%%%%%%%%%%%%%%%%%%%%%%%%%%%%%%%%%%%%%%%%%%%%%%%%%%%%%%%%%%%%%%%%%%%%%%%%%%%%%%%%%%%%%%%%%%%%%%%%%%%%%%%%%%%%%%%%%%%%%%%%%%%%%%%%%%%%%%%%%%%%%%%%%%%%%%%%%%%%%%%%%%%%%%%%%%%%%%%%%%%%%%%%%%%%%%%%%%%%%%%%%%%%%%%%%%%%%%%%%%%%%%%%%%%%%%%%%%%%%
\usepackage{amsfonts,slashed}
%\usepackage{arydshln}
%\usepackage[greek,english]{babel}
%\usepackage{cancel}
%\usepackage{upgreek}
\usepackage{float}
\usepackage{cite}
%\usepackage{url}
\usepackage{latexsym}
\usepackage{amsfonts}
\usepackage{amsmath}
\usepackage{epsfig}
%\usepackage{isomath}
\usepackage{latexsym,amssymb}
\usepackage{amsmath,amssymb,amsthm}
\usepackage{hyperref}
%\usepackage{fontenc}
\usepackage{graphicx,tikz}
\usepackage{multirow}
\setcounter{MaxMatrixCols}{13}
%\usepackage{xfrac}
%\usepackage[T1]{fontenc}
\usepackage{textcomp}
%\usepackage{lmodern}
\usepackage{xcolor}
\usepackage{mathtools}
\usepackage{stmaryrd}
%%%%%%%%% Packages to show breakings %%%%%%%%%
\usepackage{enumerate}
\usepackage[shortlabels]{enumitem}

\usepackage{xspace}
\usepackage{mathrsfs}


%  PDF specials
%\newif\ifpdf
\usepackage{ifpdf}
\ifx\pdfoutput\undefined
\pdffalse
\usepackage{cite}
\else
\pdfoutput=1
\pdftrue
%\usepackage[pdftex]{hyperref}
\pdfcompresslevel=9
\fi
\DeclareFontFamily{OMS}{rsfs}{\skewchar\font'60}
\DeclareFontShape{OMS}{rsfs}{m}{n}{<-5>rsfs5 <5-7>rsfs7 <7->rsfs10 }{}
\DeclareSymbolFont{rsfs}{OMS}{rsfs}{m}{n}
\DeclareSymbolFontAlphabet{\Scr}{rsfs}
\newcommand{\RRB}[2]{\makebox[#1]{$\left.\rule{0pt}{#2}\right)$}}
\newcommand{\LRB}[2]{\makebox[#1]{$\left(\rule{0pt}{#2}\right.$}}
\newcommand{\RSB}[2]{\makebox[#1]{$\left.\rule{0pt}{#2}\right]$}}
\newcommand{\LSB}[2]{\makebox[#1]{$\left[\rule{0pt}{#2}\right.$}}
\newcommand{\gr}[1]{\begin{greek}#1\end{greek}}
\setlength{\parskip}{0pt} \setlength{\parindent}{0.5cm}
\setcounter{footnote}{0}
\renewcommand{\theequation}{\thesection.\arabic{equation}}
\numberwithin{equation}{section}
\def\be{\begin{equation}}
	\def\ee{\end{equation}}
\def\ba{\begin{array}}
	\def\ea{\end{array}}
\newcommand{\ho}[1]{$\, ^{#1}$}
\newcommand{\hoch}[1]{$\, ^{#1}$}
\newcommand{\bea}{\begin{eqnarray}}
	\newcommand{\eea}{\end{eqnarray}}
\newcommand{\ds}{\displaystyle}
\newcommand{\BNA}{\makebox[0.8em]{\tiny$b\neq a$}}
\newcommand{\notEq}[2]{\makebox[0.8em]{\tiny$#1\neq #2$}}
\newcommand{\ph}{\alpha}
\newcommand{\scal}{z}
\newcommand{\bscal}{\bar\scal}
\newcommand{\scalC}{\pi}
%%%%%%%%%%%%%%%%% formatting
\textwidth 165mm \textheight 220mm \topmargin 0pt \oddsidemargin 2mm
%%% macros
\newcommand{\rmg}[1]{\mbox{{\small \textgreek{#1}}}}
\newcommand{\rmgg}[1]{\mathrm{#1}}
\newcommand{\ft}[2]{{\textstyle\frac{#1}{#2}}}


\newcommand{\uA}{{\underline{A}}}
\newcommand{\uB}{{\underline{B}}}
\newcommand{\uC}{{\underline{C}}}
\newcommand{\uD}{{\underline{D}}}
\newcommand{\uE}{{\underline{E}}}
\newcommand{\uF}{{\underline{F}}}
\newcommand{\uG}{{\underline{G}}}
\newcommand{\uM}{{\underline{M}}}
\newcommand{\uN}{{\underline{N}}}
\newcommand{\uP}{{\underline{P}}}
\newcommand{\dom}{{\dot{m}}}
\newcommand{\don}{{\dot{n}}}
\newcommand{\dop}{{\dot{p}}}
\newcommand{\doq}{{\dot{q}}}
\newcommand{\dor}{{\dot{r}}}
\newcommand{\ua}{{\overline{\alpha}}}
\newcommand{\ub}{{\overline{\beta}}}
\newcommand{\uc}{{\overline{\gamma}}}
\newcommand{\ude}{{\overline{\delta}}}
\newcommand{\AI}{\left(A^{-1}(\eta)\right)}
\newcommand{\ff}[1]{\mathfrak{#1}}
\newcommand{\fA}{\ff{A}}
\newcommand{\fB}{\ff{B}}
\newcommand{\FF}{{\cal F}}
\newcommand{\FG}{{\cal G}}
\newcommand{\ov}[1]{\overline{#1}}

%
%\newcommand{\hc}{{\rm h.c.}}
%\newcommand{\bbox}{\lower.2ex\hbox{$\Box$}}
%% Formatting of group names
\newcommand{\un}[1]{\mathrm{SU}( #1 )}
\newcommand{\ON}[1]{\mathrm{O}( #1 )}
\newcommand{\SU}[1]{\mathrm{SU}( #1 )}
\newcommand{\SL}[1]{\mathrm{SL}( #1 )}
\newcommand{\GL}[1]{\mathrm{GL}( #1 )}
\newcommand{\SO}[1]{\mathrm{SO}( #1 )}
\newcommand{\UO}{\mathrm{U}(1)}
\newcommand{\U}[1]{\mathrm{U}(#1)}
\newcommand{\Spin}[1]{\mathrm{Spin}(#1)}
\newcommand{\EE}{\ensuremath{E_{8(8)}}\xspace}
\newcommand{\USp}[1]{\mathrm{USp}(#1)}
\newcommand{\Un}[1]{\mathrm{U}(#1)}
\newcommand{\Gt}{\mathrm{G}_{2(2)}}
\newcommand{\Ff}{\mathrm{F}_{4(4)}}
\newcommand{\En}[1]{E_{#1(#1)}}
\newcommand{\Gst}{G_S}
\newcommand{\Msc}{{\cal M}_{\rm scalar}}
\newcommand{\SUc}[2]{\mathrm{S}(\U{#1}\times\U{#2})}

\newcommand{\Com}[2]{\mathrm{Com}_{#1}(#2)}
\newcommand{\SLA}{\SL{8}_{\rm IIA}}
\newcommand{\SLB}{\SL{8}_{\rm IIB}}

%% Formatting Lie algebra names
\newcommand{\su}[1]{\mathfrak{su}(#1)}
\newcommand{\so}[1]{\mathfrak{so}(#1)}
\newcommand{\usp}[1]{\mathfrak{usp}(#1)}
\newcommand{\gt}{\mathfrak{G}_2}
\newcommand{\fF}{\mathfrak{F}_4}
\newcommand{\uu}{\mathfrak{u}(1)}

\newcommand{\ud}[1]{\underline{#1}}

%% Group theory stuff
\newcommand{\mbf}[1]{\mathbf{#1}}
\newcommand{\mbfb}[1]{\mathbf{\left(#1\right)}}
%\newcommand{\obf}[1]{\overline{\mathbf{#1}}}
\newcommand{\obf}[1]{\mbf{\overline{\mathbf{#1}}}}
\newcommand{\trep}[1]{(\mbf{#1})}
\newcommand{\PR}[1]{(\mathbb{P}_{#1})}

\newcommand{\Leib}[2]{\llbracket #1,#2 \rrbracket}

%\newcommand{}
\newcommand{\+}{\oplus}
\newcommand{\cc}{\text{c.c.}}

%%%%%%%% generalised stuff
\newcommand{\gL}{\mathcal{L}}
\newcommand{\gM}{\mathcal{M}}
\newcommand{\tw}{{\cal U}}
\newcommand{\kr}[2]{\delta^{#1}_{#2}}
\newcommand{\cU}{{\cal U}}
\newcommand{\UI}{\left(U^{-1}\right)}
\newcommand{\flt}[1]{\underline{#1}}
\newcommand{\fl}[1]{\ov{#1}}
\newcommand{\cY}{{\cal Y}}
\newcommand{\vg}{\mathring{g}}
\newcommand{\cP}{{\cal P}}
\newcommand{\cV}{{\cal V}}
\newcommand{\cVI}{\left(\cV^{-1}\right)}
\newcommand{\cA}{\mathcal{A}}
\newcommand{\cB}{\mathcal{B}}
\newcommand{\cC}{\mathcal{C}}
\newcommand{\dg}{\vert g \vert}

%%%%%%%% math operators
\DeclareMathOperator{\tr}{Tr}

% ---- MATHS COMMANDS ----
\newcommand{\M}{\ensuremath{\mathcal{M}}\xspace}
\newcommand{\A}{\ensuremath{\mathcal{A}}\xspace}
\renewcommand{\d}{\ensuremath{\mathrm{d}}\xspace}

%% Comments
\newcommand{\EM}[1]{\textcolor{red}{#1}}
\newcommand{\MG}[1]{\textcolor{blue}{#1}}
\definecolor{darkorange}{RGB}{166, 88, 0}
\newcommand{\CE}[1]{\textcolor{darkorange}{#1}}
\newcommand{\CG}{\Com{G}{\EE}}
\newcommand{\cJ}{{\cal J}}

%%%%%%%%%%%%%%%%%%%%%%
\begin{document}
	
\begin{titlepage}
	\vfill
	\begin{flushright}
		HU-EP-23/XX
	\end{flushright}
	
	\vfill
	
	
	\begin{center}
		\baselineskip=16pt
		{\Large \bf Adding fluxes to consistent truncations: \\ IIB supergravity on ${\rm AdS}_3 \times S^3 \times S^3 \times S^1$}
		\vskip 2cm
		{\large \bf Camille Eloy$^a$\footnote{\tt camille.eloy@vub.be}, Michele Galli$^b$\footnote{\tt michele.galli@physik.hu-berlin.de},  Emanuel Malek$^b$\footnote{\tt emanuel.malek@physik.hu-berlin.de}}
		\vskip .6cm
		{\it $^a$ Theoretische Natuurkunde, Vrije Universiteit Brussel, and the Interational Solvay, \\
			Institutes, Pleinlaan 2, B-1050 Brussels, Belgium}
		\\ \ \\
		{\it $^b$ Institut f\"ur Physik, Humboldt-Universit\"at zu Berlin, \\IRIS Geb\"aude, Zum Gro\ss en Windkanal 2, 12489 Berlin, Germany \\ \ \\}
		\vskip 1cm
	\end{center}
	\vfill
		
	\begin{abstract}
		We use $\EE$ Exceptional Field Theory to construct the consistent truncation of IIB supergravity on $S^3 \times S^3 \times S^1$ to maximal 3-dimensional ${\cal N}=16$ gauged supergravity containing the ${\cal N}=(4,4)$ AdS$_3$ vacuum. We explain how to achieve this by adding a 7-form flux to the $S^1$ reduction of the dyonic $\En{7}$ truncation on $S^3 \times S^3$ previously constructed in the literature. Our truncation Ansatz includes, in addition to the ${\cal N}=(4,4)$ vacuum, a host of moduli breaking some or all of the supersymmetries. We explicitly construct the uplift of a subset of these to construct new supersymmetric and non-supersymmetric AdS$_3$ vacua of IIB string theory, and show how the moduli space of AdS$_3$ vacua of six-dimensional gauged supergravity studied in \cite{Eloy:2021fhc} is compactified upon lifting to 10 dimensions. Along the way, we also derive the form of 3-dimensional ${\cal N}=16$ gauged supergravity in terms of the embedding tensor and rule out a 10-/11-dimensional origin of some 3-dimensional gauged supergravities.
	\end{abstract}
	\vskip 0cm
		
	\vfill
		
	\setcounter{footnote}{0} 
		
\end{titlepage}
	
\tableofcontents
	
\newpage

\EM{To do:
\begin{itemize}
	\item Think about title - mention new AdS$_3$ vacua? Lifting AdS$_3$ vacua?
	\item Check the $\SL{n}$ equations
	\item Move 3-d gauged SUGRA to somewhere else??
	\item Construct 10-parameter uplift
	\item Complex geometry of moduli space?
	\item Compare with classification known in literature, e.g. in IIA (and IIB -- Jerome's story)
\end{itemize}
}

\section{Introduction}
Consistent truncations are a powerful technique that simplifies the dynamics of 10-/11-dimensional supergravity by focusing on a restricted subset of fields. The key principle behind a consistent truncation is to truncate to a subsector of fields, such that all solutions of the truncated theory correspond to solutions of the original 10-/11-dimensional supergravity theory. By considering only a, typically finite, subset of fields, consistent truncations provide a powerful tool to find complicated new 10-/11-dimensional supergravity solutions and to study their deformations. Consistent truncations have proven particularly powerful in the AdS/CFT correspondence, since all well-understood AdS vacua of string theory do not admit scale-separation and, thus, cannot be studied using the usual tools of lower-dimensional effective theories.

Exceptional Field Theory (ExFT) is a reformulation of 10-/11-dimensional supergravity that unifies the metric and flux degrees of freedom and thereby makes manifest an exceptional symmetry group\footnote{For the purposes of this paper, we do not draw a distinction between ExFT and Exceptional Generalised Geometry, since these agree when the ``section condition'' is solved, which we will always assume here.}. Over the last decade, this has proven an extremely useful tool for constructing consistent truncations, leading to a number of new examples preserving various amounts of supersymmetry \cite{Hohm:2014qga,Lee:2014mla,Malek:2015hma,Baguet:2015sma,Baguet:2015iou,Lee:2015xga,Malek:2016bpu,Ciceri:2016dmd,Cassani:2016ncu,Inverso:2016eet,Malek:2017cle,Malek:2017njj,Malek:2018zcz,Malek:2019ucd,Cassani:2019vcl,Malek:2020jsa,Cassani:2020cod}. However, consistent truncations to three dimensions have, until recently \cite{Galli:2022idq}, remained largely unexplored. One reason is that the $\EE$ ExFT requires modifications for its local symmetry structure, i.e. the generalised Lie derivative, to close into an algebra \cite{Hohm:2014fxa,Cederwall:2015ica}. Another is that there are no maximally supersymmetric AdS$_3$ vacua, i.e. preserving 32 supercharges, of string theory, leaving no natural candidate for constructing a consistent truncation with vacua. Indeed, while the consistent truncations of 11-dimensional supergravity on $S^7$ and $S^4$ and of 10-dimensional supergravity on $S^5$ contain maximally supersymmetric AdS vacua and are captured by a universal Ansatz in ExFT \cite{Hohm:2014qga,Lee:2014mla}, the analogous $S^8$ truncation of 11-dimensional supergravity does not exist, while there are two $S^7$ truncations of 10-dimensional supergravity but neither contains a maximally symmetric vacuum\cite{Fischbacher:2003yw,Galli:2022idq}.

Nonetheless, there are half-maximal, i.e. ${\cal N}=(4,4)$, supersymmetric AdS$_3$ vacua of string theory that are extremely intriguing. These are the AdS$_3 \times S^3 \times S^3 \times S^1$ and AdS$_3 \times S^3 \times T^4$ of IIB string theory (realising the ``small'' and ``large'' ${\cal N}=(4,4)$ superconformal symmetries), and, unlike in higher dimensions, can be supported by pure NS-NS flux. Not only does this mean that there are also heterotic versions of these vacua (preserving ${\cal N}=(4,0)$ supersymmetry), but also that they can be readily studied via the string worldsheet CFT \cite{Maldacena:2000hw,Eberhardt:2018ouy,Eberhardt:2019niq,Eberhardt:2019ywk}. Moreover, the string sigma model in these vacua is also integrable \cite{}. Finally, the AdS$_3$/CFT$_2$ correspondence has seen much progress recently, with many new holographic duals being established \cite{}.

Here we will focus on the ${\cal N}=(4,4)$ AdS$_3 \times S^3 \times S^3 \times S^1$ vacuum of IIB string theory and show that it admits a consistent truncation to a maximal gauged supergravity, which was first studied in \cite{Hohm:2005ui}. All vacua of this theory break at least half the supersymmetries, reflecting the absence of a maximally supersymmetric AdS$_3$ vacuum in string theory. Key to constructing this consistent truncation is to add new fluxes to the $S^1$ reduction of the consistent truncation of IIB supergravity on $S^3 \times S^3$ \cite{Inverso:2016eet}.

Using our consistent truncation, we will show that the AdS$_3 \times S^3 \times S^3 \times S^1$ has a rich moduli space of symmetry- and supersymmetry-breaking deformations, including some that are analogous to the ``flat deformations'' studied for AdS$_4 \times S^5 \times S^1$ \cite{Guarino:2020gfe,Guarino:2021hrc,Giambrone:2021zvp,Giambrone:2021wsm}. While some of these deformations preserve some amounts of supersymmetry, others break all supersymmetries, yet, surprisingly, at least a subset of vacua continue to be perturbatively stable within IIB supergravity, similar to \cite{Giambrone:2021wsm}. Our work also provides an uplift of 6-dimensional supergravity on $S^3$ studied in \cite{Eloy:2021fhc} to IIB string theory. As we will show, some deformations that appear non-compact in 6-dimensional supergravity \cite{Eloy:2021fhc} are compactified within the full 10-dimensional supergravity theory.

Another technical result of our work is the derivation of the potential of 3-dimensional gauged supergravity in terms of the embedding tensor, which was previously only expressed in terms of fermion shift matrices. As a by-product, we find that a constraint that must be obeyed by the embedding tensor of any 3-dimensional gauged supergravity that can be uplifted to 10-/11-dimensional supergravity. As we show, this allows us to prove a higher-dimensional origin of some 3-dimensional gaugings.

The outline of our paper is as follows. We begin with a review of $\EE$ ExFT in section \ref{s:Review} and how to derive consistent truncations in this formalism in \ref{s:TruncationReview}. In section \ref{s:Potential} we derive the potential of 3-dimensional gauged supergravity in terms of the embedding tensor and prove the lack of higher-dimensional origin of some 3-dimensional theories. Then, we show how to add fluxes to consistent truncations in \ref{s:AddFlux}, allowing us to construct the consistent truncation of IIB supergravity on $S^3 \times S^3 \times S^1$. Using this truncation, we study the moduli space of the AdS$_3$ vacua in section \ref{s:Moduli}, before concluding in section \ref{s:Conclusions}.

\section{Review of \texorpdfstring{\EE}{E8(8)} exceptional field theory} \label{s:Review}

The \EE exceptional field theory, first constructed in ref.~\cite{Hohm:2014fxa}, is an \EE duality-covariant formulation of type II and 11d supergravities. It is defined on a set of $3+248$ coordinates made of three-dimensonal external coordinates $\{x^{\mu}\}$ and internal coordinates $\{Y^{M}\}$ in the 248-dimensional adjoint representation of \EE. The dependance of the fields on these coordinates is constrained by the ``section constraints''\footnote{We use the notation $\otimes$ to indicate that both derivatives may act on different functions.}
\begin{equation} \label{eq:sectionconstraints}
	\begin{cases}
		\eta^{MN}\partial_{M}\otimes\partial_{N} = 0\,,\\
		f^{MN}{}_{P}\,\partial_{M}\otimes\partial_{N} = 0\,,\\
		(\mathbb{P}_{3875})_{MN}{}^{KL}\partial_{K}\otimes\partial_{L} = 0\,,
	\end{cases}
\end{equation}
where $f_{MN}{}^{P}$ are the structure constants of \EE, $\eta_{MN} = \tfrac{1}{60}\,f_{MK}{}^{L}f_{NL}{}^{K}$ its Cartan-Killing metric and $(\mathbb{P}_{3875})_{MN}{}^{KL}$ is the projector on the representation $\mathbf{3875}$:
\begin{equation}
	(\mathbb{P}_{3875})_{MN}{}^{KL} = \frac{1}{7}\, \delta_{(M}{}^{K}\delta_{N)}{}^{L}-\frac{1}{56}\,\eta_{MN}\eta^{KL}-\frac{1}{14}\,f^{P}{}_{M}{}^{(K}f_{PN}{}^{L)}\,.
\end{equation}
Here and in the following, \EE indices are raised and lowered by the Cartan-Killing metric $\eta_{MN}$. The section constraints~\eqref{eq:sectionconstraints} ensure that the fields depends only on the 7- or 8-dimensional physical internal coordinates embeded in $Y^{M}$.

The theory describes the dynamics of the following bosonic fields:
\begin{equation}
	\left\{g_{\mu\nu}, \M_{MN}, A_{\mu}{}^{M}, B_{\mu\,M}\right\}\,,
\end{equation}
with $g_{\mu\nu}$ the 3-dimensinal external metric, $\M_{MN}$ the generalized metric parametrizing the coset space $\EE/\SO{16}$ and the gauge fields $A_{\mu}{}^{M}$ and $B_{\mu\,M}$. It is a gauge theory, invariant under the generalized Lie derivative of parameters $\Upsilon=(\Lambda^{M},\Sigma_{M})$, whose action on a vector $V^{M}$ of weight $\lambda$ is given by
\begin{equation} \label{eq:genlie}
	{\cal L}_{\Upsilon} V^{M} = \Lambda^{N}\partial_{N}V^{M}-60\,(\mathbb{P}_{248})^{M}{}_{N}{}^{K}{}_{L}\,V^{N}\partial_{K}\Lambda^{L} + \lambda\, V^{M}\partial_{N}\Lambda^{N}+f^{MN}{}_{K}\Sigma_{N}V^{K}\,,
\end{equation}
with $(\mathbb{P}_{248})^{M}{}_{N}{}^{K}{}_{L}=(1/60)\,f^{M}{}_{NP}f^{PK}{}_{L}$ the projector on the adjoint representation. These transformations are well-defined (in particular, they close into an algebra) only if the parameters $\Sigma_{M}$ and the fields $B_{\mu\, M}$ are covariantly constrained: they have to satisfy algebraic constraints similar to eq.~\eqref{eq:sectionconstraints} and be compatibile with the partial derivatives. We require that
\begin{equation} \label{eq:sectionconstraint2}
	\begin{cases}
		\eta^{MN}C_{M}\otimes C'_{N} = 0\,,\\
		f^{MN}{}_{P}\,C_{M}\otimes C'_{N} = 0\,,\\
		(\mathbb{P}_{3875})_{MN}{}^{KL}C_{K}\otimes C'_{L} = 0\,,
	\end{cases}
	\forall\  C_{M}, C'_{M} \in \{\partial_{M}, \Sigma_{M}, B_{\mu\,M}\}\,.
\end{equation}

\paragraph{}
The bosonic action of \EE exceptional field theory is invariant under the transformations~\eqref{eq:genlie} and has the expression
\begin{equation} \label{eq:ExFTaction}
	S_{\rm ExFT} = \int d^{3}x\,d^{248}Y\,\sqrt{\vert g\vert} \bigg(\widehat{R}+\frac{1}{240}\,D_{\mu}\M_{MN}D^{\mu}\M^{MN}+\mathscr{L}_{\rm int}+\frac{1}{\sqrt{\vert g\vert}}\,\mathscr{L}_{\rm CS}\bigg)\,.
\end{equation}
$\widehat{R}$ is the \EE-covariantised Ricci scalar and the covariant derivative is defined as
\begin{equation}
	D_{\mu} = \partial_{\mu} - {\cal L}_{(A_{\mu}, B_{\mu})}\,.
\end{equation}
$\mathscr{L}_{\rm int}$ is a potential term depending only on internal derivatives, explicitly
\begin{equation} \label{eq:ExFTLint}
	\begin{aligned}
		\mathscr{L}_{\rm int} &= \frac{1}{240}\,\M^{MN}\partial_{M}\M^{KL}\partial_{N}\M_{KL} - \frac{1}{2}\,\M^{MN}\partial_{M}\M^{KL}\partial_{L}\M_{NK} \\
		&\quad -\frac{1}{7200}\,f^{NQ}{}_{P}f^{MS}{}_{R}\,\M^{PK}\partial_{M}\M_{QK}\,\M^{RL}\partial_{N}\M_{SL}\\
		&\quad +\frac{1}{2}\,g^{-1}\partial_{M}g\,\partial_{N}\M^{MN} + \frac{1}{4}\,\M^{MN}\,g^{-2}\partial_{M}g\,\partial_{N}g + \frac{1}{4}\,\M^{MN}\,\partial_{M}g_{\mu\nu}\,\partial_{N}g^{\mu\nu}\,.
	\end{aligned}
\end{equation}
Finally, $\mathscr{L}_{\rm CS}$ is a Chern-Simons term required to impose the on-shell duality between scalar and vector fields, given by
\begin{equation}
	\begin{aligned}
		\mathscr{L}_{\rm CS} &= \frac{1}{2}\,\varepsilon^{\mu\nu\rho}\,\bigg(F_{\mu\nu}{}^{M}B_{\rho\,M}-f_{KL}{}^{N}\,\partial_{\mu}A_{\nu}{}^{K}\partial_{N}A_{\rho}{}^{L}-\frac{2}{3}\,f^{N}{}_{KL}\,\partial_{M}\partial_{N}A_{\mu}{}^{K}A_{\nu}{}^{M}A_{\rho}{}^{L}\\
		&\quad -\frac{1}{3}\,f_{MKL}\,f^{KP}{}_{Q}\,f^{LR}{}_{S}\,A_{\mu}{}^{M}\partial_{P}A_{\nu}{}^{Q}\partial_{R}A_{\rho}{}^{S}\bigg),
	\end{aligned}
\end{equation}
where $F_{\mu\nu}{}^{M}$ is the covariant field strength of $A_{\mu}{}^{M}$ (see eq. (2.26) of ref.~\cite{Hohm:2014fxa}).

\section{Consistent truncations to 3-dimensional \texorpdfstring{${\cal N}=16$}{N=16} gauged supergravity} \label{s:TruncationReview}
The \EE exceptional field theory is well suited to the construction of consistent truncation of type~II and 11d supergravities to three-dimensional maximal supergravity. These truncations arise as generalised Scherk-Schwarz reductions, described by an \EE-valued twist matrix $U_{M}{}^{\fl{M}}$ and a scale factor $\rho$. The truncation Ansätze are as follows~\cite{Galli:2022idq}:
\begin{equation} \label{eq:GSSansatz}
	\begin{aligned}
		g_{\mu\nu}(x,Y)&=\rho(Y)^{-2}\,\mathring{g}_{\mu\nu}(x)\,,\\
		\M_{MN}(x,Y)&=U_{M}{}^{\fl{M}}(Y)\,U_{N}{}^{\fl{N}}(Y)\,\M_{\fl{MN}}(x)\,,\\
		A_{\mu}{}^{M}(x,Y)=&\rho(Y)^{-1}\,\UI_{\fl{M}}{}^{M}(Y)\,\A_{\mu}{}^{\fl{M}}(x)\,,\\
		B_{\mu\,M}(x,Y)&=\Sigma_{\fl{M}\,M}(Y)\,\A_{\mu}{}^{\fl{M}}(x)\,,
	\end{aligned}
\end{equation}
with
\begin{equation}
	\Sigma_{\fl{M}\,M}=\frac{\rho^{-1}}{60}\,f_{\fl{M}}{}^{\fl{PQ}}\,\UI_{\fl{P}P}\,\partial_{M}\UI_{\fl{Q}}{}^{P}\,.
\end{equation}
The fields with flat indices $\fl{M}$ (in the $\mathbf{248}$ representation of \EE) belong to the three-dimensional theory; all the dependence on the internal manifold is factored out in $U_{\fl{M}}{}^{M}$ and $\rho$. Note that with the definition of $\Sigma_{\fl{M}\,M}$ the condition~\eqref{eq:sectionconstraint2} is automatically satisfied.

The Ansätze~\eqref{eq:GSSansatz} describe a \textit{consistent} truncation if the following condition is satisfied:
\begin{equation} \label{eq:GenLeibniz}
	\gL_{\ff{U}_{\fl{M}}} {\cU}_{\fl{N}}{}^M = X_{\fl{M}\fl{N}}{}^{\fl{P}}\, \cU_{\fl{P}}{}^M\,,
\end{equation}
with $\cU_{\fl{M}}{}^M = \rho^{-1} \UI_{\fl{M}}{}^M$, $\ff{U}_{\fl{M}} = (\cU_{\fl{M}},\,\Sigma_{\fl{M}})$ and $X_{\fl{M}\fl{N}}{}^{\fl{P}}$ a constant tensor. This condition ensures that the factorised form of the Ansätze~\eqref{eq:GSSansatz} is preserved by the generalized Lie derivative, \textit{e.g.}
\begin{equation} \label{eq:GSSexample}
	\begin{aligned}
		D_{\mu}\M(x,Y) &= U_{M}{}^{\fl{M}}(Y)\,U_{N}{}^{\fl{N}}(Y)\,{\cal D}_{\mu}\M_{\fl{MN}}(x)\,,\\
		{\cal L}_{(A_{\mu},B_{\nu})}A_{\mu}(x,Y) &= \rho(Y)^{-1}\,\UI_{\fl{M}}{}^{M}(Y)\,\llbracket\A_{\mu},\A_{\nu}\rrbracket^{\fl{M}}(x)\,,
	\end{aligned}
\end{equation}
where
\begin{equation} \label{eq:3dCovDeriv}
	{\cal D}_{\mu}\M_{\fl{MN}} = \partial_{\mu}\M_{\fl{MN}}-2\,\A_{\mu}{}^{\fl{P}}X_{\fl{P}(\fl{M})}{}^{\fl{Q}}\M_{\fl{N})\fl{Q}}\,, \quad {\rm and} \quad \llbracket\A_{\mu},\A_{\nu}\rrbracket^{\fl{M}}=X_{\fl{PQ}}{}^{\fl{M}}\A_{\mu}{}^{\fl{P}}\A_{\nu}^{\fl{Q}} \,.
\end{equation}

Thus, for a given twist matrix $U_{M}{}^{\fl{M}}$ satisfying the consistency condition~\eqref{eq:GenLeibniz}, the action~\eqref{eq:ExFTaction} reduces to three-dimensional ${\cal N}=16$ gauged supergravity. The constant tensor $X_{\fl{M}\fl{N}}{}^{\fl{P}}$ plays the role of the embedding tensor of the three-dimensional gauged supergravity, as is clear from its appearance in \eqref{eq:3dCovDeriv} in the gauge-covariant derivative and the gauge algebra of the vector fields. From eq. \eqref{eq:GenLeibniz}, it has the following expression in terms of the twist matrix and $\rho$
\begin{equation}
	\begin{aligned}
		X_{\fl{MN}}{}^{\fl{P}}&=-\rho^{-1}\,\Gamma_{\fl{MN}}{}^{\fl{P}} + \rho^{-1}\,f^{\fl{P}}{}_{\fl{NQ}}\,f^{\fl{QK}}{}_{\fl{L}}\,\Gamma_{\fl{KM}}{}^{\fl{L}}-\frac{1}{60}\,\rho^{-1}\,f^{\fl{PK}}{}_{\fl{N}}\,f_{\fl{ML}}{}^{\fl{Q}}\,\Gamma_{\fl{KQ}}{}^{\fl{L}}\\
		&\quad-\frac{1}{2}\,\rho^{-1}\,f^{\fl{P}}{}_{\fl{NQ}}\,f^{\fl{QK}}{}_{\fl{M}}\,\Gamma_{\fl{RK}}{}^{\fl{R}}+\left(\delta_{\fl{M}}{}^{\fl{K}}\delta_{\fl{N}}{}^{\fl{P}}-\frac{1}{2}\,f_{\fl{M}}{}^{\fl{LK}}f_{\fl{NL}}{}^{\fl{P}}\right)\,\xi_{\fl{K}}\,.
	\end{aligned}
\end{equation}
Here, we defined the ${\rm E}_{8(8)}$ current $\Gamma_{\fl{MN}}{}^{\fl{P}}=U_{\fl{M}}{}^{K}U_{\fl{N}}{}^{L}\partial_{K}U^{\fl{P}}{}_{L}$ and the trombone gauging \EM{Give this in terms of $\Gamma_{\fl{M}\fl{N}}$ for later use?}
\begin{equation} \label{eq:Trombone}
	\xi_{\fl{M}} = 2\,U_{\fl{M}}{}^{N}\partial_{N}\rho^{-1}+\rho^{-1}\,\partial_{N}U_{\fl{M}}{}^{N} \,.
\end{equation}
In the following, we consider only gaugings with $\xi_{\fl{M}}=0$. The embedding tensor is then most nicely expressed once projected on the adjoint representation \CE{(Keep expression with projector on 3875?)} \EM{I think it's useful}
\begin{equation} \label{eq:symembeddingtensor}
	\begin{aligned}
		X_{\fl{MN}} = -\frac{1}{60}\,X_{\fl{MP}}{}^{\fl{Q}}\,f_{\fl{NQ}}{}^{\fl{P}} &= 2\,\rho^{-1}\,\Gamma_{(\fl{MN})} - \rho^{-1}\,\Gamma_{\fl{P}(\fl{M}}{}^{\fl{Q}}\,f_{\fl{N})\fl{Q}}{}^{\fl{P}}\\
		&=14\,\rho^{-1}\,(\mathbb{P}_{3875})_{\fl{MN}}{}^{\fl{PQ}}\,\Gamma_{\fl{PQ}}+\frac{1}{4}\,\rho^{-1}\,\eta_{\fl{MN}}\,\Gamma_{\fl{P}}{}^{\fl{P}} \,,
	\end{aligned}
\end{equation}
with the projection of the current $\Gamma_{\fl{MN}} = \tfrac{1}{60}\,\Gamma_{\fl{MP}}{}^{\fl{Q}}\,f_{\fl{NQ}}{}^{\fl{P}}$.

The three-dimensional action follows from inserting the Ansätze~\eqref{eq:GSSansatz} in the ExFT action~\eqref{eq:ExFTaction}. The expressions of the kinetic and Chern-Simons terms result immediately from eq.~\eqref{eq:GSSexample} \CE{(A word on the Ricci scalar?)} \EM{Yes, should briefly mention what happens}
\begin{equation}
	\begin{aligned}
		\mathscr{L}_{\rm kin} &= \frac{1}{240}\,D_{\mu}\M_{MN}D^{\mu}\M^{MN} \underset{\rm gSS}{=} \frac{\rho^{2}}{240}\,{\cal D}_{\mu}\M_{\fl{MN}}{\cal D}^{\mu}\M^{\fl{MN}}, \\
		\mathscr{L}_{\rm CS} &\underset{\rm gSS}{=} -\rho^{-1}\,\varepsilon^{\mu\nu\rho}\,X_{\fl{MN}}\,\A_{\mu}{}^{\fl{M}} \left(\partial_{\nu}\A_{\rho}{}^{\fl{N}}-\frac{1}{3}\,\llbracket\A_{\nu},\A_{\rho}\rrbracket^{\fl{N}}\right)\,.
	\end{aligned}
\end{equation}

However, the case of the potential term $\mathscr{L}_{\rm int}$, leading to a potential $V$ for the three-dimensional scalars $\M_{\fl{MN}}$, is more subtle. We expect $V$ to be quadratic in the embedding tensor, which makes the identification of $X_{\fl{MN}}$ more complicated. Moreover, the potential of three-dimensional ${\cal N}=16$ gauged supergravity in terms of the embedding tensor is unknown, with currently the potential only expressed in terms of the fermion shift matrices of the gauged supergravity. Thus, in the following, we will follow the truncation procedure carefully and thereby construct the potential of the three-dimensional ${\cal N}=16$ gauged supergravity in terms of the embedding tensor.

\section{Deriving the potential of ${\cal N}=16$ gauged supergravity} \label{s:Potential}
Here we will use the $\EE$ ExFT Lagrangian and the consistent truncation Ansatz \eqref{eq:GSSansatz} to derive the potential of three-dimensional ${\cal N}=16$ gauged supergravity in terms of the embedding tensor. Not only is this an interesting application of ExFT, relying on purely bosonic considerations and bypassing the usual construction of the gauged supergravity potential using supersymmetry\footnote{Note that the same strategy was recently used to impressively derive the potential of maximal two-dimensional gauged supergravity \cite{Bossard:2022wvi}, where fermions are extremely poorly understood. As a result, in that case, the bosonic $\En{9}$ ExFT \cite{Bossard:2018utw,Bossard:2021jix} provides the only currently accessible route for computing the gauged supergravity potential.}, but the potential is crucial for finding vacua of the three-dimensional theory and uplifting these in the later parts of this paper.

In order to derive the potential, we adopt the following strategy. We want $\mathscr{L}_{\rm int}$ \eqref{eq:ExFTLint} to reduce to the embedding tensor squared upon inserting the generalised Scherk-Schwarz Ans\"{a}tze \eqref{eq:GSSansatz}. However, as in higher dimensions, e.g. \cite{Berman:2012uy,Musaev:2013rq,Blair:2014zba}, this does not have to match identically, but only up to total derivative terms and terms which violate the section condition. The possible boundary terms are
\begin{equation} \label{eq:BoundaryTerms}
	\frac{1}{\sqrt{\dg}} \partial_M \left(\sqrt{\dg} \partial_N \gM^{MN} \right) \quad \text{and} \quad \frac{1}{\sqrt{\dg}} \partial_M \left( \gM^{MN} \partial_N \sqrt{\dg}\right) \,.
\end{equation}
However, since the ${\cal N}=16$ supergravity potential will be given by terms quadratic in $X_{\fl{MN}}{}^{\fl{P}}$, we do not want the boundary terms to involve any double derivative terms. Therefore, the two boundary terms \eqref{eq:BoundaryTerms} can only appear via the combination
\begin{equation} \label{eq:BoundaryTermCombination}
	\frac{1}{\sqrt{\dg}}\,\partial_{M}\Big(\sqrt{\dg}\,\partial_{N}\M^{MN}+\frac{4}{3}\,\M^{MN}\partial_{N}\sqrt{\dg}\Big) \,.
\end{equation}

Now, we first insert the generalized Sherk-Schwarz Ansätze \eqref{eq:GSSansatz} into the expression of $\mathscr{L}_{\rm int}$ \eqref{eq:ExFTLint} with the additional total derivative \eqref{eq:BoundaryTermCombination} \CE{(Need to motivate the total derivative? Mention the need for section constraint? Factor $\sqrt{\vert g\vert}$ to avoid carrying $\rho$ in $V$?)}
\begin{equation} \label{eq:Lintexpansion}
	\begin{aligned}
		\mathscr{L}_{\rm int} &+ \frac{a}{\sqrt{\vert g\vert}}\,\partial_{M}\Big(\sqrt{\vert g\vert}\,\partial_{N}\M^{MN}+\frac{4}{3}\,\M^{MN}\partial_{N}\sqrt{\vert g\vert}\Big)\\
		&\underset{\rm gSS}{=} -\frac{1}{2}\,\M^{\fl{MN}}\,\Gamma_{\fl{MK}}\Gamma_{\fl{N}}{}^{\fl{K}} -\M^{\fl{MN}}\,\Gamma_{\fl{MK}}\Gamma^{\fl{K}}{}_{\fl{N}} -\frac{1}{2}\,\Gamma_{\fl{MN}}\Gamma^{\fl{NM}} - \frac{1}{2}\,\M^{\fl{MN}}\M^{\fl{KL}}\,\Gamma_{\fl{MK}}\Gamma_{\fl{NL}} \\
		&\ \quad - \frac{1}{2}\,\M^{\fl{MN}}\M^{\fl{KL}}\,\Gamma_{\fl{MK}}\Gamma_{\fl{LN}} +\frac{3}{2}\left(a-1\right)\,\M^{\fl{MN}}\Gamma_{\fl{MN}}{}^{\fl{K}}\Gamma_{\fl{LK}}{}^{\fl{L}}-\frac{a}{2}\,\M^{\fl{MN}}\Gamma_{\fl{KM}}{}^{\fl{K}}\Gamma_{\fl{LN}}{}^{\fl{L}}\\
		&\ \quad + \left(\frac{1}{2}-a\right)\,\M^{\fl{MN}}\Gamma_{\fl{KM}}{}^{\fl{L}}\Gamma_{\fl{LN}}{}^{\fl{K}}+\,\M^{\fl{MN}}\Gamma_{\fl{MK}}{}^{\fl{L}}\Gamma_{\fl{LN}}{}^{\fl{K}}-\frac{1}{2}\,\M^{\fl{MN}}\M^{\fl{KL}}f_{\fl{QL}}{}^{\fl{P}}\Gamma_{\fl{PN}}{}^{\fl{Q}}\Gamma_{\fl{MK}}\,.
	\end{aligned}
\end{equation}

Secondly, knowing that the ${\cal N}=16$ potential must be quadratic in $X_{\fl{MN}}$, we consider the most general quadratic function in $X_{\fl{MN}}$ and develop it in \EE currents using eq.~\eqref{eq:symembeddingtensor}. \CE{(Sign convention between $V$ and ${\mathscr{L}}_{\rm int}$? $+{\mathscr{L}}_{\rm int}$ in eq.~\eqref{eq:ExFTaction} but $-V$ in eq.~\eqref{eq:ExFTactiongSS}.)} Note that in higher dimensions, there exist only two quadratic combinations of the embedding tensor
\begin{equation}
	X_{\fl{M}\fl{N}}{}^{\fl{P}} X_{\fl{Q}\fl{P}}{}^{\fl{N}} \gM^{\fl{MQ}} \qquad \text{and} \qquad X_{\fl{M}\fl{N}}{}^{\fl{P}} X_{\fl{Q}\fl{R}}{}^{\fl{S}} \gM^{\fl{MQ}} \gM^{\fl{N}\fl{R}} \gM_{\fl{P}\fl{S}} \,.
\end{equation}
However, in three dimensions, because the vector fields transform in the adjoint representation of $\EE$, we can write two additional terms
\begin{equation} \label{eq:ExtraX2Terms}
	X_{\fl{MN}} X_{\fl{PQ}}\, \eta^{\fl{MP}} \eta^{\fl{NQ}} \qquad \text{and} \qquad X_{\fl{MN}} X_{\fl{PQ}}\, \eta^{\fl{MN}} \eta^{\fl{PQ}} \,,
\end{equation}
where the second term corresponds to the square of the singlet part of the embedding tensor.

It turns out that for gaugings that arise from consistent truncations, these two terms \eqref{eq:ExtraX2Terms} can be related to the square of the trombone tensor. We can square eq.~\eqref{eq:symembeddingtensor} to derive an expression for $X_{\ov{MN}}X^{\ov{MN}}$ in terms of the current $\Gamma$. We find 
\begin{equation}\label{eq:XXwitheta}
	X_{\ov{MN}}X^{\ov{MN}} = 21 \rho^{-2}\, \Gamma_{\ov{MN}}\, \Gamma^{\ov{NM}} + 19 \rho^{-2}\left(\Gamma_{\ov M}{}^{\ov M}\right)^2\,.
\end{equation}
Similarly, squaring the trombone tensor $\xi$ gives \EM{Should we give \eqref{eq:Trombone} in terms of $\Gamma$?}, we find
\begin{equation}\label{eq:xixiwitheta}
	\xi_{\ov M}\,\xi^{\ov M}=\rho^{-2}\left(\Gamma_{\ov M}{}^{\ov M}\Gamma_{\ov N}{}^{\ov N}+\Gamma_{\ov{MN}}\Gamma^{\ov{NM}}\right) \,.
\end{equation}
We can now deduce
\begin{equation}
	X_{\ov{MN}} \,X^{\ov{MN}}=21	\xi_{\ov M} \, \xi^{\ov M} - 2 \rho^{-2}\left(\Gamma_{\ov M}{}^{\ov M}\right)^2 \,,
\end{equation}
which can be expressed entirely in terms of $X_{\fl{M}\fl{N}}$ after tracing eq.~\eqref{eq:symembeddingtensor}
\begin{equation}\label{eq:reltrX}
	X_{\ov{MN}}\,X^{\ov{MN}} = 21\, \xi_{\ov M}\, \xi^{\ov M} - \frac1{1922}\left(X_{\ov M}{}^{\ov M}\right)^2 \,. 
\end{equation}
Thus, \eqref{eq:reltrX} is a constraint on the embedding tensor of three-dimensional supergravity which must be satisfied in order for it to have an uplift to 10-/11-dimensional supergravity. Note that this constraint is not implied by the quadratic constraint. To see this, simply observe that the theory with $\En{8}$ gauging has $X_{\ov{MN}}=\eta_{\ov{MN}}$ and $\xi_{\fl{M}} = 0$, violating \eqref{eq:reltrX}.

Since we are focusing on gaugings with vanishing trombone $\xi_{\fl{M}} = 0$, \eqref{eq:reltrX} implies that the two terms in \eqref{eq:ExtraX2Terms} are proportional and we only need to consider one of these terms in the gauged supergravity potential. Thus, the most general Ansatz for the supergravity potential, for gaugings that have an uplift to 10-/11-dimensional supergravity\footnote{If we do not require the three-dimensional gauged supergravity to arise from a consistent truncation, \eqref{eq:reltrX} may be violated and we need to include both terms of \eqref{eq:ExtraX2Terms}. However, here we are not interested in such gaugings.}, is
\begin{equation} \label{eq:Vexpansion}
	\begin{aligned}
		-V &= X_{\fl{MN}}X_{\fl{PQ}}\left(\alpha\,\M^{\fl{MP}}\M^{\fl{NQ}}+\beta\,\M^{\fl{MP}}\eta^{\fl{NQ}}+\delta\,\eta^{\fl{MP}}\eta^{\fl{NQ}}
		%+\gamma\,\eta^{\fl{MN}}\eta^{\fl{PQ}}
		\right) \\
		&\underset{\rm gSS}{=} \rho^{-2}\bigg(14\,\alpha\,\Gamma_{\fl{MN}}\,\Gamma_{\fl{PQ}}\,\M^{\fl{MP}}\M^{\fl{NQ}}+14\,\alpha\,\Gamma_{\fl{MN}}\,\Gamma_{\fl{PQ}}\,\M^{\fl{MQ}}\M^{\fl{NP}}\\
		&\ \quad +2\,\beta\,\M^{\fl{MN}}\,\Gamma_{\fl{MK}}\Gamma^{\fl{K}}{}_{\fl{N}}+\beta\,\M^{\fl{MN}}\,\Gamma_{\fl{MK}}\,\Gamma_{\fl{N}}{}^{\fl{K}}+\left(-12\,\alpha+2\,\delta
		%-(62)^2\,\gamma
		\right)\,\Gamma_{\fl{MN}}\,\Gamma^{\fl{MN}}\\
		&\ \quad -2\,\beta\,\M^{\fl{MN}}\Gamma_{\fl{MK}}{}^{\fl{L}}\,\Gamma_{\fl{LN}}{}^{\fl{K}}+\beta\,\M^{\fl{MN}}\Gamma_{\fl{KM}}{}^{\fl{L}}\,\Gamma_{\fl{LN}}{}^{\fl{K}}+\left(7\,\alpha+\frac{\beta}{2}\right)\,\M^{\fl{MN}}\Gamma_{\fl{KM}}{}^{\fl{K}}\,\Gamma_{\fl{LN}}{}^{\fl{L}}\\
		&\ \quad +14\,\alpha\,\M^{\fl{MN}}\M^{\fl{KL}}f_{\fl{QL}}{}^{\fl{P}}\Gamma_{\fl{PN}}{}^{\fl{Q}}\,\Gamma_{\fl{MK}}\bigg)\,.
	\end{aligned}
\end{equation}
Finally, comparing eq.~\eqref{eq:Lintexpansion} and~\eqref{eq:Vexpansion}, the parameters are fixed to
\begin{equation}
	\alpha = \frac{\rho^{2}}{28}\,,\quad \beta = \frac{\rho^{2}}{2}\,,\quad a=1\,,\quad \delta=\frac{13}{28}\,\rho^{2}\,. %, \quad \gamma=0.
\end{equation}
%\CE{Freedom for $\delta$ and $\gamma$: $2\delta-(62)^2\gamma=\rho^{2}(1/2+3/7)$.}
%\EM{MG and EM are confused about this -- we need to chat all together}

\CE{Some formula used to derive the previous equations:
\begin{equation}
	\begin{aligned}
		\M^{\fl{MP}}\M^{\fl{NQ}}f_{\fl{PQ}}{}^{\fl{K}}&=-f^{\fl{MN}}{}_{\fl{L}}\,\M^{\fl{LK}},\\
		{\rm Ad\ invariance:}\quad 2\,\Gamma_{\fl{K}[\fl{M}}{}^{\fl{L}}\,f_{\fl{N}]\fl{L}}{}^{\fl{P}}&= -f_{\fl{MN}}{}^{\fl{L}}\,\Gamma_{\fl{KL}}{}^{\fl{P}}\,,\\
		{\rm No\ trombone:}\quad \UI_{\fl{M}}{}^{M}\partial_{M}\ln(\rho)&= -\frac{1}{2}\,\Gamma_{\fl{KM}}{}^{\fl{K}},\\
		{\rm Eq.~\eqref{eq:sectionconstraints}}\ \implies \ f^{\fl{K}}{}_{\fl{MP}}f^{\fl{PL}}{}_{\fl{N}}\,\partial_{\fl{K}}\otimes\partial_{\fl{L}} &= 2\,\partial_{(\fl{M}}\otimes\partial_{\fl{N})}\,.
	\end{aligned}
\end{equation}
}

% Consider a twist matrix $U_{\bar M}{}^{M}$ and a scale factor $\rho$. A consistent truncation requires that the objects
% \begin{equation} \label{eq:DoubleU}
% 	\ff{U}_{\fl{M}} = (\cU_{\fl{M}},\,\Sigma_{\fl{M}}) \,, \qquad \Sigma_{\fl{M}\,M} = \frac{1}{60} \rho^{-1} f_{\fl{M}}{}^{\fl{N}\fl{P}}\, U_{\fl{N}\,P} \partial_M U_{\fl{P}}{}^{P} \,, \qquad \cU_{\fl{M}}{}^M = \rho^{-1} U_{\fl{M}}{}^M \,,
% \end{equation}
% satisfy the differential condition
% \begin{equation} \label{eq:GenLeibniz}
% 	\gL_{\ff{U}_{\fl{M}}} {\cU}_{\fl{N}}{}^M = X_{\fl{M}\fl{N}}{}^{\fl{P}}\, \cU_{\fl{P}}{}^M \,,
% \end{equation}


With the generalized Scherk-Schwarz Ansätze~\eqref{eq:GSSansatz}, the action~\eqref{eq:ExFTaction} becomes
\begin{equation} \label{eq:ExFTactiongSS}
	\begin{aligned}
		S_{\rm ExFT} \underset{\rm gSS}{=} \int\d^{248}Y\,\rho^{-1}\int \d^{3}x & \bigg[\sqrt{\vert\mathring{g}\vert}\bigg(\mathring{R}+\frac{1}{240}\,{\cal D}_{\mu}\M_{\fl{MN}}{\cal D}^{\mu}\M^{\fl{MN}}-V\bigg) \\
		&-\varepsilon^{\mu\nu\rho}\,X_{\fl{MN}}\,\A_{\mu}{}^{\fl{M}} \left(\partial_{\nu}\A_{\rho}{}^{\fl{N}}-\frac{1}{3}\,\llbracket\A_{\nu},\A_{\rho}\rrbracket^{\fl{N}}\right)\bigg],
	\end{aligned}
\end{equation}
with the potential%\footnote{To obtain this potential, one has to add the total derivative $-\partial_{M}\left(\sqrt{\vert g\vert}\,\partial_{N}\M^{MN}+(4/3)\,\M^{MN}\partial_{N}\sqrt{\vert g\vert}\right)$ to $\sqrt{\vert g\vert}\,\mathscr{L}_{\rm int}$.}
\begin{equation}
	V = X_{\fl{MN}}\,X_{\fl{PQ}}\left(\frac{1}{28}\,\M^{\fl{MP}}\M^{\fl{NQ}}+\frac{1}{2}\,\M^{\fl{MP}}\eta^{\fl{NQ}}+\frac{13}{28}\,\eta^{\fl{MP}}\eta^{\fl{NQ}}\right).
\end{equation}
\CE{(A word on the 3d linear and quadratic constraints?)}

%\subsection{Relating $\tr( X^2)$ and $(\tr X) ^2$ using $\xi$}
%One can square eq.~\eqref{eq:symembeddingtensor} to derive an expression for $X_{\ov{MN}}X^{\ov{MN}}$ in terms of the current $\Gamma$, we find 
%\begin{equation}\label{eq:XXwitheta}
%	X_{\ov{MN}}X^{\ov{MN}}=21\rho^{-2}\Gamma_{\ov{MN}}\Gamma^{\ov{NM}}+19\rho^{-2}\left(\Gamma_{\ov M}{}^{\ov M}\right)^2.
%\end{equation}
%Similarly, squaring the trombone tensor $\xi$ gives 
%\begin{equation}\label{eq:xixiwitheta}
%	\xi_{\ov M}\xi^{\ov M}=\rho^{-2}\left(\Gamma_{\ov M}{}^{\ov M}\Gamma_{\ov N}{}^{\ov N}+\Gamma_{\ov{MN}}\Gamma^{\ov{NM}}\right),
%\end{equation}
%we can then deduce
%\begin{equation}
%	X_{\ov{MN}}X^{\ov{MN}}=21	\xi_{\ov M}\xi^{\ov M}-2\rho^{-2}\left(\Gamma_{\ov M}{}^{\ov M}\right)^2
%\end{equation}
%which can be expressed entirely in terms of $X$ after tracing eq.~\eqref{eq:symembeddingtensor}:
%\begin{equation}\label{eq:reltrX}
%		X_{\ov{MN}}X^{\ov{MN}}=21	\xi_{\ov M}\xi^{\ov M}-\frac1{1922}\left(X_{\ov M}{}^{\ov M}\right)^2. 
%\end{equation}
%Whenever $\xi=0$, eq.~\eqref{eq:reltrX} is a constraint on the embedding tensor of the resulting three dimensional supergravity which does not arise from the quadratic constraint. To see this, simply observe that the theory with $\En{8}$ gauging constructed in \MG{REFERENCE}, has $X_{\ov{MN}}=\eta_{\ov{MN}}$, violating~\eqref{eq:reltrX}. \\
%Since we used the Scherk-Schwarz Ansatz in the calculation, we deduce that this must in fact be a condition satisfied by any theory that comes from type II or 11 dimensional supergravity. For example \MG{$\SO{4}\times \SO{4}$ gauging ...}

\subsection{Ruling out gaugings}

\section{Adding fluxes to consistent truncations} \label{s:AddFlux}
In order to construct the consistent truncation around the ${\cal N}=(4,4)$ AdS$_3 \times S^3 \times S^3 \times S^1$ vacuum of IIB supergravity, a promising starting point is the dyonic $S^3 \times S^3$ truncation of IIB supergravity constructed in \cite{Inverso:2016eet}, and further reducing this on $S^1$. However, this will not give the correct AdS$_3$ vacuum since the truncation is missing the 7-form flux on $S^3 \times S^3 \times S^1$ that is required. Indeed, the 3-dimensional gauged supergravity that would be obtained as discussed way \CE{(?)} does not have any AdS$_3$ vacua. We can remedy this situation by defining a new consistent truncation by adding a 7-form flux to the one obtained from the $S^3 \times S^3$ reduction constructed in $\En{7}$ ExFT.

This motivates the following question, given a consistent truncation, i.e. $\cU_{\fl{M}}{}^M$, satisfying \eqref{eq:GenLeibniz}, when can we add a new flux component of string theory to the compactification to obtain a new consistent truncation? Adding a new flux component to the truncation is equivalent to twisting the generalised frame as follows
\begin{equation} \label{eq:Twisting}
	\cU_{\fl{M}}{}^M \longrightarrow \cU'_{\fl{M}}{}^M = \cU_{\fl{M}}{}^N \exp(C)_N{}^M \,,
\end{equation}
where $C$ denotes the $\EE$ generator corresponding to the potential we want to add to the compactification. The effect of the twist \eqref{eq:Twisting} is that the generalised Lie derivative of $\cU'$ satisfies
\begin{equation}
	\gL_{\ff{U}'_{\fl{M}}} \cU'_{\fl{N}}{}^M = \left( \gL_{\ff{U}_{\fl{M}}} \cU_{\fl{N}}{}^N + F_{PQ}{}^N\, \cU_{\fl{M}}{}^P\, \cU_{\fl{N}}{}^Q \right) \exp(C)_N{}^M \,,
\end{equation}
where $F_{MN}{}^P$ is a tensor in the $\mathbf{1} \oplus \mathbf{248} \oplus \mathbf{3875}$ of $\EE$, i.e. the same representation as the embedding tensor, corresponding to the field strength of the potential $C$ in \eqref{eq:Twisting}. Using \eqref{eq:GenLeibniz}, we now have
\begin{equation}
	\gL_{\ff{U}'_{\fl{M}}} \cU'_{\fl{N}}{}^M = \left( X_{\fl{M}\fl{N}}{}^{\fl{P}} + \rho^{-1}\, F_{\fl{M}\fl{N}}{}^{\fl{P}} \right) \cU'_{\fl{P}}{}^M \,,
\end{equation}
where we defined
\begin{equation} \label{eq:FlatF}
	F_{\fl{M}\fl{N}}{}^{\fl{P}} \equiv \UI_{\fl{M}}{}^M\, \UI_{\fl{N}}{}^N\, U_P{}^{\fl{P}}\, F_{MN}{}^P \,,
\end{equation}
and $X_{\fl{M}\fl{N}}{}^{\fl{P}}$ is already constant. Therefore, we have a consistent truncation if and only iff $\rho^{-1}\, F_{\fl{M}\fl{N}}{}^{\fl{P}}$ is constant.

A particularly simple way of having constant $\rho^{-1}\, F_{\fl{M}\fl{N}}{}^{\fl{P}}$ is to consider fluxes which are stabilised by the twist matrix $U_M{}^{\fl{M}}$. The twist matrix typically only lives in a subgroup $G \subset \EE$. Therefore, in this case, we can simply tune the $G$-singlet components of the flux $F_{MN}{}^P$ to be proportional to $\rho$ to obtain a new consistent truncation.

\subsection{Adding fluxes to the $S^3$ truncation of 6-dimensional supergravity} \label{s:S3Flux}
Before dealing with the $S^3 \times S^3 \times S^1$ truncation, let us demonstrate this methodology for the consistent truncation of ${\cal N}=(1,1)$ 6-dimensional supergravity on $S^3$, which was constructed in \cite{Eloy:2021fhc}. As discussed in \cite{Eloy:2021fhc}, the consistent truncation of ${\cal N}=(1,1)$ 6-dimensional supergravity to 3-dimensional half-maximal gauged supergravity can be described using the $\SO{8,4}$ ExFT \cite{Hohm:2017wtr,Samtleben:2019zrh}. On the other hand, the consistent truncation on $S^3$ is conveniently described by twist matrices living in $\SL{4} \simeq \SO{3,3}$ \cite{Lee:2014mla,Hohm:2014qga,Baguet:2015iou} \EM{I think the twist matrix of \cite{Baguet:2015iou,Lee:2014mla,Hohm:2014qga} coincide in the $\SL{4}$ case, right? Camille, do you agree?} with the embedding $\SO{3,3} \subset \SO{4,4} \subset \SO{8,4}$. However, the twist matrix in \cite{Eloy:2021fhc} differs from this $\SO{3,3}$ twist matrix \cite{Lee:2014mla,Hohm:2014qga,Baguet:2015iou} by an additional parameter, $\lambda$, which gives rise to an external 3-form flux or, equivalently, a new internal 3-form flux.

\subsubsection{The $S^3$ twist matrix}
Since it will play a role throughout this paper, we will begin by giving the $\SL{4} \simeq \SO{3,3}$ twist matrix that describes the consistent truncation on $S^3$. Here, we give a slightly different, but equivalent, form of the twist matrix constructed in \cite{Lee:2014mla,Hohm:2014qga,Baguet:2015iou} for the consistent truncation on $S^n$.

\EM{Maybe even move this $\SL{n+1}$ ExFT to the review section?}
We use an $\SL{n+1}$ ExFT, like in \cite{Lee:2014mla}, which encodes an $n$-dimensional metric and volume-form flux, i.e. Freund-Rubin compactifications, and is thus a natural description of $S^n$ compactifications. The $\SL{n+1}$ ExFT formally has coordinates in the antisymmetric representation of $\SL{n+1}$, $y^{IJ} = - y^{JI}$, with $I = 1, \ldots, n+1$, and similarly, generalised vector fields transform in the antisymmetric representation of $\SL{n+1}$. The generalised Lie derivative on a generalised tensor in the fundamental $V^I$ of weight $\lambda$ is given by
\begin{equation} \label{eq:SLnGenLie}
	\gL_\Lambda V^I = \frac12 \Lambda^{JK} \partial_{JK} V^{I} - V^J \partial_{JK} \Lambda^{IK} + \left( \frac{\lambda}{2} + \frac{1}{n+1} \right) V^I \partial_{JK} \Lambda^{JK} \,,
\end{equation}
and similarly for other generalised tensors. Closure of the generalised Lie derivative \eqref{eq:SLnGenLie} requires the section condition
\begin{equation} \label{eq:SLnSC}
	\partial_{[IJ} \otimes \partial_{KL]} = 0 \,,
\end{equation}
which restricts the dependence of all fields to a subset of physical coordinates.

We are interested in the maximal solutions of the section condition \eqref{eq:SLnSC} which preserve a $\SL{n} \subset \SL{n+1}$ subgroup. Under this decomposition $\SL{n+1} \rightarrow \SL{n}$ with $V^I = \left( V^i,\, V^0 \right)$, where $i = 1, \ldots, n$, we solve the section condition by having physical coordinates $y^{i0}$ on the $n$-dimensional manifold $M$, i.e. $\partial_{ij} = 0$ for all fields with only $\partial_{i0} \neq 0$. Correspondingly, the generalised tangent bundle, whose sections are in the antisymmetric representation of $\SL{n+1}$ and carry weight $\frac{n-3}{n+1}$, decomposes as
\begin{equation} \label{eq:SLnE}
	E = TM \oplus \Lambda^{n-2} T^*M \,,
\end{equation}
Note that for the case of interest to us, $n=3$, \eqref{eq:SLnE} reduces to
\begin{equation}
	E = TM \oplus T^*M \,,
\end{equation}
and its fibres transform in the $\mathbf{6}$ of $\SL{4} \simeq \SO{3,3}$. It is convenient to also introduce the generalised bundle with fibres in the anti-fundamental of $\SL{n+1}$, which is, similarly, given by
\begin{equation}
	N = T^*M \oplus \Lambda^{n} T^*M \,.
\end{equation}
Sections of $N$ are generalised tensors transforming in the anti-fundamental of $\SL{n+1}$ and carrying weight $\frac{2}{n+1}$. For $V, V' \in \Gamma(E)$ and $W \in \Gamma(N)$, the generalised Lie derivative \eqref{eq:SLnGenLie} reduces to
\begin{equation}
	\begin{split}
		\gL_{V} V' &= [v, v'] + L_{v} \omega'_{(n-2)} - \imath_{v'} d\omega_{(n-2)} \,, \\
		\gL_{V} W &= L_v \omega_{(1)} + L_v \omega_{(n)} - \EM{\alpha} \omega_{(1)} \wedge d\omega_{(n-2)} \,,
	\end{split}
\end{equation}
\EM{(I think $\alpha = 1$ but not 100\% sure)}
where we write $V = v + \omega_{(n-2)}$, $V' = v' + \omega'_{(n-2)}$ and $W = \omega_{(1)} + \omega_{(n)}$ as a formal of vectors and $p$-forms, and $L$ denotes the ordinary Lie derivative and $[v,v']$ the ordinary Lie bracket between vector fields $v$ and $v'$.

We can now describe the $S^n$ consistent truncation using the $\SL{n+1}$ ExFT above. Let $Y^I$, $I = 1, \ldots, n+1$ be the embedding coordinates of $S^{n} \subset \mathbb{R}^{n+1}$, such that $Y^I\, Y_I = 1$. On $S^n$, we can define a parallelisation of $N$ using the generalised frame \EM{Check sign of A! Although I'm pretty confident about it.}
\begin{equation}\label{eq:SnFrame}
	\cU^{\fl{I}} = dY^{\fl{I}} + Y^{\fl{I}}\, vol_{S^n} - A \wedge dY^{\fl{I}} \,,
\end{equation}
where $A$ is an $(n-1)$-potential with field strength $dA = (n-1) vol_{S^n}$, and
\begin{equation}
	vol_{S^n} = \frac{1}{n!}\epsilon_{I_1 I_2 \ldots I_{n+1}} Y^{I_1}\, dY^{I_2} \wedge \ldots \wedge dY^{I_{n+1}} \,,
\end{equation}
is the volume form on $S^n$. The frame in \eqref{eq:SnFrame} has the important property that the $n+1$ generalised tensors are nowhere vanishing since $dY^I = 0$ only when $Y^I = 1$.

In a local basis, \eqref{eq:SnFrame} gives us a $\GL{n+1}$ matrix, $\cU_I{}^{\fl{I}}$, whose determinant allows us to define the scalar density
\begin{equation} \label{eq:rho}
	\rho = \left(\det \cU_I{}^{\fl{I}}\right)^{-1/2} = \left(\det \mathring{g}\right)^{-1/2} \,,
\end{equation}
with $\mathring{g}$ the round metric on $S^n$. Using \eqref{eq:SnFrame} and \eqref{eq:rho} we can define the $\SL{n}$ twist matrix
\begin{equation} \label{eq:SLnFrame}
	U_I{}^{\fl{I}} = \rho^{2/(n+1)}\, \cU_I{}^{\fl{I}} \,.
\end{equation}
and hence a generalised frame for $E$ or a generalised bundle in any other rep of $\SL{n+1}$.

Evaluating \eqref{eq:SLnFrame} and \eqref{eq:rho} on the northern hemisphere $Y^I = \left( y^i,\, \sqrt{1-y^i y_i}\right)$, $i = 1, \ldots, n$ and using the gauge choice for the potential $A$
\begin{equation}
	A_{ij} = \epsilon_{ijk}\, y^k (1 + K(v)) \,,
\end{equation}
with $v = y^i\, y^i$ and $\epsilon_{ijk}$ the volume form on $S^3$, we precisely recover the $\SL{4}$ twist matrix of \cite{Hohm:2014qga}, i.e.
\begin{equation} \label{eq:SLnTwist}
	\UI_{\fl{I}}{}^I = \begin{pmatrix}
		\left(1-v\right)^{-1/4} \left( \delta_i{}^j + y_i\, y^j K(v) \right) & y_i \left(1-v\right)^{1/4} \\
		y^j \left(1-v\right)^{1/4} K(v) & \left(1-v\right)^{3/4}
	\end{pmatrix} \,.
\end{equation}
Moreover, the generalised frame for $E$ precisely coincides with the twist matrix in \cite{Lee:2014mla,Baguet:2015iou}. \EM{Give it explicitly?}

\EM{To do: Should mention Leibniz parallelisability somewhere and the embedding tensor that arises}

\EM{I think the sign conventions are:
	\begin{equation}
		\begin{split}
			\cU^{\fl{I}} &= dY^{\fl{I}} + Y^{\fl{I}}\, vol_{S^n} + dY^{\fl{I}} \wedge A \,, \\
			\gL_{V} W &= L_v \omega_{(1)} + L_v \omega_{(n)} + \omega_{(1)} \wedge d\omega_{(n-2)} \,, \\
			\gL_{\cU_{\fl{I}\fl{J}}} \cU^{\fl{K}} &= - X_{\fl{I}\fl{J},\fl{L}}{}^{\fl{K}}\, \cU^{\fl{L}} \,, \\
			X_{\fl{I}\fl{J},\fl{L}}{}^{\fl{K}} &= 2\, \delta_{\fl{L}[\fl{I}} \delta_{\fl{J}]}^{\fl{K}} \,.
		\end{split}
	\end{equation}
For the twist matrix in the antisymmetric rep, I find
\begin{equation}
	\cU_{[\fl{I}} \cU_{\fl{J}]} = v_{\fl{J}\fl{I}} + \star \left( dY_{\fl{I}} \wedge dY_{\fl{J}} \right) + \left(-1\right)^{n-1} \imath_{v_{\fl{J}\fl{I}}} A \,,
\end{equation}
where the vector part is $V^{i\,0}$, the $(n-2)$-form part is $V^{ij} = \frac{1}{(n-2)!} \epsilon^{k_1 \ldots k_{n-2} ij} \lambda_{k_{1} \ldots k_{n-2}}$ and the coordinates are identified as $Y^{i\,0} = y^i$. We can probably match the sign of the vector part of \cite{Baguet:2015iou} by taking the vector part to be $V^{0\,i}$ and also changing the coordinates to be $Y^{0\,i} = y^i$. We should probably check this carefully since it's useful to write up properly.
}

\subsubsection{Adding flux}
To demonstrate our methodology, we will now show how the parameter $\lambda$ introduced in \cite{Eloy:2021fhc} can be obtained by the twisting procedure outlined in \ref{s:AddFlux}. Let us decompose $\SO{4,4} \rightarrow \SO{3,3} \times \SO{1,1}$, such that
\begin{equation}
	\begin{split}
		\mathbf{8} &\rightarrow \mathbf{6_0} \oplus \mathbf{1_2} \oplus \mathbf{1_{-2}} \,, \\
		\mathbf{28} &\rightarrow \mathbf{15_0} \oplus \mathbf{6_2} \oplus \mathbf{6_{-2}} \oplus \mathbf{1_0} \,.
	\end{split}
\end{equation}
Correspondingly, we write a $\SO{4,4}$ vector $V^M$ as
\begin{equation}
	V^M = \left( V^A,\, V^z,\, V^{\bar{z}} \right) \,,
\end{equation}
where $A = 1, \ldots, 6$ labels the vector of $\SO{3,3}$ and $z$, $\bar{z}$ label the $\mathbf{1_2}$ and $\mathbf{1_{-2}}$, respectively.

We denote by $U_A{}^{\ov{A}}$ the $\SO{3,3}$ twist matrix corresponding to the $S^3$ truncation constructed in \cite{Lee:2014mla,Baguet:2015iou} and obtained from \eqref{eq:SLnFrame}, \eqref{eq:rho}, \eqref{eq:SLnTwist}. Then, we can add a 3-form potential by twisting with the $\SO{4,4}$ generator
\begin{equation} \label{eq:C2S3example}
	(e^C)^{MN} = C_A \left( t^A \right)^{MN} \,,
\end{equation}
with $C_A$ an element of the $\mathbf{6_{2}}$, and $\left( t^A \right)^{MN}$ the corresponding to $\SO{4,4}$ generator whose only non-zero component is
\begin{equation}
	\left(t^A\right)^{Bz} = \eta^{AB} \,.
\end{equation}
Here $\eta_{MN}$ and $\eta_{AB}$ are the $\SO{4,4}$ and $\SO{3,3}$ invariant metrics, respectively, and are used to raise/lower the corresponding vector indices. Decomposing with respect to the geometric $\SL{3} \times \mathbb{R}^+$ of the $S^3$,  the coordinates on $S^3$ live in the $Y^A = \left( y^i,\, y_i \right)$, $i = 1, 2, 3$, while $C_A = \left( C_i,\, C^i \right)$ naturally contains a 2-form $C^i$. The field strength $H_3 = \partial^A C_A = \partial_i C^i$ is a singlet of $\SO{3,3}$ and thus we see that $F_{AB}{}^C$ in \eqref{eq:FlatF} is constant \CE{(what about the $\rho$ factor?)} and we obtain a consistent truncation with a new 3-form flux. Evaluating the twist matrix
\begin{equation}
	U'_M{}^{\ov{M}} = (e^C)_M{}^N  U_N{}^{\ov{M}} \,,
\end{equation}
explicitly using \eqref{eq:C2S3example} and setting $C^i = \lambda\, \xi^i$ in the notation of \cite{Eloy:2021fhc}, we obtain precisely the twist matrix used in \cite{Eloy:2021fhc} for the consistent truncation of 6-dimensional ${\cal N}=(1,1)$ supergravity on $S^3$ with $H_3$ flux. \EM{Give precise form of $C^i$ such that we recover $\lambda$?}

%\EM{Spell out this as an example of the method}\\
%In fact, the consistent truncation of AdS$_3 \times S^3$ constructed in \cite{Eloy:2021fhc} can be understood exactly as arising in this way: it utilises the generalised parallelisability of $S^3$ with $U_M{}^A \in \SL{4} \simeq \SO{3,3}$ twist matrices but then adds a volume flux to the $S^3$ which is a singlet of $\SL{4}$ to obtain a new twist matrix via \eqref{eq:Twisting} which lives in $\SO{4,4}$. \EM{Should mention that this is a subsector of the $\SO{4,8}$ setup there?}
We will now follow this same procedure in the next section to obtain the consistent truncation of IIB supergravity on $S^3 \times S^3 \times S^1$ with 7-form flux.

\subsection{Adding flux to the $S^3 \times S^3 \times S^1$ truncation of IIB supergravity}
Our strategy for constructing the consistent truncation on $S^3 \times S^3 \times S^1$ with 7-form flux is to embed the $S^3 \times S^3$ truncation of $\En{7}$ ExFT \cite{Inverso:2017lrz} into $\EE \rightarrow \En{7} \times \SL{2}$, as described in \cite{Galli:2022idq} to generate the consistent truncation on $S^3 \times S^3 \times S^1$ without 7-form flux. We then add a 7-form flux to this truncation as outlined above.

\subsubsection{The $S^3 \times S^3$ truncation} \label{s:S3S3}
Let us first review the dyonic $S^3 \times S^3$ truncation of IIB supergravity \cite{Inverso:2017lrz}, which forms the starting point of our construction. The key step in the construction of \cite{Inverso:2017lrz} is that we can use the $\SL{4}$ generalised frame \eqref{eq:SLnFrame} to form a generalised Leibniz parallelisation of $\En{7}$ via the embedding
\begin{equation}
	\En{7} \rightarrow \SL{8} \rightarrow \SL{4} \times \SL{4} \times \mathbb{R}^+ \,.
\end{equation}
The fundamental of $\En{7}$ then decomposes as
\begin{equation}
	\mathbf{56} \rightarrow \mathbf{28} \oplus \overline{\mathbf{28}} \rightarrow \left[ \mathbf{\left(6,1\right)_2} \oplus \mathbf{\left(1,6\right)_{-2}} \oplus \mathbf{\left(4,4\right)_0} \right] \oplus \left[ \mathbf{\left(6,1\right)_{-2}} \oplus \mathbf{\left(1,6\right)_{2}} \oplus \mathbf{\left(\overline{4},\overline{4}\right)_0} \right] \,,
\end{equation}
where the first square brackets denote the branching of the $\mathbf{28}$ under $\SL{4} \times \SL{4} \times \mathbb{R}^+$ and the second square brackets that of the $\mathbf{\overline{28}}$. Crucially, for the generalised Leibniz condition \eqref{eq:GenLeibniz} to hold, the coordinates and generalised vector fields corresponding to the two $\SL{4}$'s must be embedded within the $\mathbf{28}$ and $\mathbf{\overline{28}}$, respectively, which we will call ``electric'' and ``magnetic'' coordinates, following \cite{Inverso:2017lrz}. Thus, we write the 56 $\En{7}$ coordinates as
\begin{equation}
	Y^M = \left( Y^{\cA\cB} ,\, Y_{\cA\cB} \right) \,,
\end{equation}
with $\cA, \cB = 1, \ldots, 8$ labelling the fundamental of $\SL{8}$, corresponding to the decomposition $\En{7} \rightarrow \SL{8}$. Following the conventions of \cite{Inverso:2017lrz}, the coordinates of the two $\SL{4}$ ExFTs are embedded as $Y^{IJ} \subset Y^{\cA\cB}$, with $I, J = 1, 2, 3, 8$, and $Y_{AB} \subset Y_{\cA\cB}$, with $A, B = 4, 5, 6, 7$. Solving the $\SL{4}$ ExFT section condition for $Y^{IJ}$ and $Y_{AB}$ guarantees a solution to the $\En{7}$ ExFT, and we choose the solution where the six physical coordinates of IIB are
\begin{equation} \label{eq:PhysicalCoords}
	y^i = Y^{i8},\, \qquad \tilde{y}_a = Y_{a7} \,, \qquad i = 1, 2, 3 \,, \qquad a = 4, 5, 6 \,.
\end{equation}

We can now use one copy of the $\SL{4}$ frame \eqref{eq:SLnFrame} for each $\SL{4}$ subgroup to construct a generalised parallelisation for the full $\En{7}$ ExFT, with an embedding tensor via \eqref{eq:GenLeibniz} \EM{given by \ldots}.

\EM{Include breaking of adjoint here?}\\
\EM{Should mention the geometric $\SL{3}$ subgroups here???}

\CE{(Add $\alpha$ here?)}

\subsubsection{The $S^3 \times S^3 \times S^1$ truncation} \label{s:S3S3S1}
We now construct the consistent truncation on $S^3 \times S^3 \times S^1$ by embedding the $\En{7}$ twist matrix corresponding to the $S^3 \times S^3$ truncation reviewed above \ref{s:S3S3} in $\EE$ via the branching $\EE \rightarrow \En{7} \times \SL{2}$, such that $\mathbf{248} \rightarrow \mathbf{\left(133,1\right)} \oplus \mathbf{\left(56,2\right)} \oplus \mathbf{\left(1,3\right)}$. As explained in \cite{Galli:2022idq}, we can construct a consistent truncation on $M \times S^1$ to 3-dimensional gauged supergravity by embedding a consistent truncation on $M$ to 4-dimensional supergravity, characterised by a $\En{7}$ twist matrix, as follows. The $\EE$ twist matrix is parameterised by the $\En{7}$ twist matrix describing the consistent truncation on $M$ and the $\SL{2}$ twist matrix given by
\begin{equation}
	v_{\ov{i}}{}^i = \begin{pmatrix}
		\sigma & 0 \\ 0 & \sigma^{-1}
	\end{pmatrix} \,,
\end{equation}
where $\sigma$ is the $\En{7}$ scalar density used to construct the $\En{7}$ truncation on $M$. Finally, the $\EE$ scalar density satisfies $\rho = \sigma^2$ to ensure a consistent truncation on $S^3 \times S^3 \times S^1$. However, we still need to add the 7-form flux to obtain the ${\cal N}=(4,4)$ AdS$_3$ vacuum we are looking for. To do this, let us first review the group theory of the $S^3 \times S^3$ truncation, since this will allow us to determine whether the 7-form flux is stabilised by the twist matrix.

\subsubsection{Adding flux}
%We now wish to add a 7-form flux to the $S^3 \times S^3 \times S^1$ truncation constructed above.
To add a 7-form flux to the above truncation, we will follow the logic laid out in section \ref{s:AddFlux}, i.e. we will investigate whether the 7-form flux is stabilised by the twist matrix on $S^3 \times S^3 \times S^1$ constructed in section \ref{s:S3S3S1}. Thus, we need to understand 
%the group theory underlying the consistent truncation on $S^3 \times S^3$ \cite{Inverso:2016eet}. The $\En{7}$ twist matrix describing this truncation is constructed from two $\SL{4}$ twist matrices, one for each $S^3$ similar to \ref{s:S3Flux}, via the decomposition
%\begin{equation}
%	\En{7} \rightarrow \SL{8} \rightarrow \SL{4} \times \SL{4} \times \mathbb{R}^+ \,,
%\end{equation}
%under which the fundamental of $\En{7}$ branches as
%\begin{equation}
%	\mathbf{56} \rightarrow \mathbf{28} \oplus \overline{\mathbf{28}} \rightarrow \left[ \mathbf{\left(6,1\right)_2} \oplus \mathbf{\left(1,6\right)_{-2}} \oplus \mathbf{\left(4,4\right)_0} \right] \oplus \left[ \mathbf{\left(6,1\right)_{-2}} \oplus \mathbf{\left(1,6\right)_{2}} \oplus \mathbf{\left(\overline{4},\overline{4}\right)_0} \right] \,,
%\end{equation}
%where the first square brackets denote the branching of the $\mathbf{28}$ under $\SL{4} \times \SL{4} \times \mathbb{R}^+$ and the second square brackets that of the $\mathbf{\overline{28}}$.
%We want to see if 
whether IIB supergravity admits a 7-form field strength that is a singlet under these two $\SL{4}$ groups. To answer this question, we pick a gauge where the 6-form potential lives entirely in $S^3 \times S^3$ but depends on the $S^1$ coordinate, $z$. Therefore, in this gauge choice, the 6-form potential corresponds to an adjoint generator of $\En{7}$, and since the $S^1$ coordinate is an $\En{7}$ singlet, this adjoint generator must be a singlet under $\SL{4} \times \SL{4} \subset \SL{8} \subset \En{7}$ in order for us to have a consistent truncation.

IIB supergravity contains an S-duality doublet of 6-forms, which we can easily identify using the decomposition $\En{7} \rightarrow \SL{6} \times \SL{2} \times \mathbb{R}^+_{\rm IIB}$, under which the $\En{7}$ adjoint decomposes as
\begin{equation} \label{eq:AdjointSL6}
	\begin{split}
		\mathbf{133} &\rightarrow \mathbf{\left(35,1\right)_0} \oplus \mathbf{\left(1,3\right)_0} \oplus \mathbf{\left(1,1\right)_0} \oplus \mathbf{\left(1,2\right)_{\pm 6}} \\
		& \quad \oplus \mathbf{\left(\overline{15},2\right)_2} \oplus \mathbf{\left(\overline{15},1\right)_{-4}} \oplus \mathbf{\left(15,2\right)_{-2}} \oplus \mathbf{\left(15,1\right)_4} \,.
	\end{split}
\end{equation}
The 6-form doublet corresponds to the $\mathbf{\left(1,2\right)}_{6}$. To understand if one of the 6-forms in the doublet are singlets under $\SL{4} \times \SL{4} \subset \En{7}$, we must decompose $\En{7}$ with respect to the common subgroup of $\SL{4} \times \SL{4}$ and $\SL{6} \times \SL{2}$, which is $\SL{3}_1 \times \SL{3}_2 \times \mathbb{R}^+ \times \mathbb{R}^+$. The two $\SL{3}$ subgroups are defined by the embedding of the physical coordinates \eqref{eq:PhysicalCoords} into $\SL{4} \times \SL{4} \subset \SL{8} \subset \En{7}$. An important for us is to correctly identify the $\mathbb{R}^+$ charges in both decompositions. We have, on the one hand,
\begin{equation} \label{eq:SL4SL4}
	\begin{split}
		\En{7} &\rightarrow \SL{8} \rightarrow \SL{4} \times \SL{4} \times \mathbb{R}^+_{\SL{8}} \\
		&\rightarrow \SL{3}_1 \times \SL{3}_2 \times \mathbb{R}^+_1 \times \mathbb{R}^+_2 \times \mathbb{R}^+_{\SL{8}} \,,
	\end{split}
\end{equation}
and, on the other,
\begin{equation} \label{eq:SL6SL2}
	\begin{split}
		\En{7} &\rightarrow \SL{6} \times \SL{2} \times \mathbb{R}^+_{\rm IIB} \\
		&\rightarrow \SL{3}_1 \times \SL{3}_2 \times \mathbb{R}^+_{\SL{2}} \times \mathbb{R}^+_{\SL{6}} \times \mathbb{R}^+_{\rm IIB} \,.
	\end{split}
\end{equation}
Here, we label the $\mathbb{R}^+$'s with a subscript that refers to the the groups they belong to. The $\mathbb{R}^+$ generators of the two decompositions are related as
\begin{equation} \label{eq:R+charges}
	\begin{split}
		\mathbb{R}^+_{\SL{6}} &= - \frac12 \left( \mathbb{R}^+_{\SL{3}_1} + \mathbb{R}^+_{\SL{3}_2} \right) \,, \\
		\mathbb{R}^+_{\SL{2}} &= \frac14 \left( \mathbb{R}^+_{\SL{3}_1} - \mathbb{R}^+_{\SL{3}_2} + \mathbb{R}^+_{\SL{8}} \right) \,, \\
		\mathbb{R}^+_{\rm IIB} &= \frac12 \left( \mathbb{R}^+_{\SL{3}_1} - \mathbb{R}^+_{\SL{3}_2} - 3\, \mathbb{R}^+_{\SL{8}} \right) \,.
	\end{split}
\end{equation}

The branching of the adjoint \eqref{eq:AdjointSL6} under $\SL{6} \times \SL{2} \times \mathbb{R}^+_{\rm IIB}$ in \eqref{eq:SL6SL2} needs to now be compared with that via $\SL{4} \times \SL{4} \times \mathbb{R}^+_{\SL{8}}$, given by
\begin{equation}
	\begin{split}
		\mathbf{133} &\rightarrow \mathbf{63} \oplus \mathbf{70} \\
		&\rightarrow \mathbf{\left(15,1\right)_0} \oplus \mathbf{\left(1,15\right)_0} \oplus \mathbf{\left(1,1\right)_0} \oplus \mathbf{\left(4,\overline{4}\right)_2} \oplus \mathbf{\left(\overline{4},4\right)_{-2}} \\
		& \quad \oplus \mathbf{\left(1,1\right)_{-4}} \oplus \mathbf{\left(1,1\right)_{4}} \oplus \mathbf{\left(4,\overline{4}\right)_{-2}} \oplus \mathbf{\left(\overline{4},4\right)_{2}} \oplus \mathbf{\left(6,6\right)_0} \,.
	\end{split}
\end{equation}
Using \eqref{eq:R+charges}, we can now identify each of the singlet generators $\left(1,1\right)_{\pm4}$ with one of the $\SL{2}$ doublet generators of charge $\mp 6$. Therefore, we find exactly one singlet $\SL{4} \times \SL{4}$ generator, which we can write as $t_{1238}$, corresponding to a 6-form potential. The other $\SL{4} \times \SL{4}$ singlet generator, $t_{4567}$, only differs by a compact generator and thus does not corresponding to a different physical field. Finally, the other elements of the $\SL{2}$ doublet of 6-form potential can be mapped to the $\mathbf{\left(4,\overline{4}\right)_{2}} \oplus \mathbf{\left(\overline{4},4\right)_{-2}}$ generators, specifically $t_7{}^8$ and $t_8{}^7$ and are clearly not $\SL{4}$ singlets.
\EM{Is this sufficiently clear? I am not so sure...}

Therefore, we can add a 7-form flux to the twist matrix using
\begin{equation}
	\cU_{\fl{M}}{}^M \longrightarrow \cU'_{\fl{M}}{}^M = \cU_{\fl{M}}{}^N \exp(C)_N{}^M \,,
\end{equation}
with $C_M{}^N = \lambda\, z\, t_{1238\,M}{}^N$ and $\lambda$ a numerical parameter corresponding to the amount of 7-form flux. \EM{Is there a density factor that needs to be added?} \CE{Yes.} \EM{The resulting embedding tensor is \ldots}

%, which transform under the subgroup $\SL{6} \times \SL{2} \times \mathbb{R}^+_{IIB} \subset \En{7}$ as $\mathbf{\left(1,2\right)_{6}} \subset \mathbf{133}$

%\EM{Describe how $C_6{}^\alpha$ sits in here}

\section{Moduli of AdS$_3 \times S^3 \times S^3 \times S^1$} \label{s:Moduli}

\subsection{Moduli in gauged supergravity}

\subsection{Moduli in 10 dimensions}
\MG{Maybe we recall what the vacuum at the origin is here and reference somebody?}
The seven form flux is unaffected by the deformations.

\paragraph{SUSY deformation}

%In Hopf coordinates:
%\begin{equation}
%	\begin{split}
%		ds^2 &= \alpha^2 \left[ ds^2_{\mathbb{CP}^1} + \kappa^2 \right] + ds^2_{\tilde{S}^3} + e^{-4\omega} d\psi^2 \,, \\
%		\kappa &= d\theta + \sigma - \frac1{\alpha} e^{-2\omega} \sqrt{\CE{-}(1-e^{2\omega})} d\psi \,, \\
%		d\sigma &= vol_{\mathbb{CP}^1} \,.
%	\end{split}
%\end{equation}

It is convenient to use the Hopf fibration to describe this, we denote by $\phi$ the coordinate on the $S^1$ fibre and by $\xi$ the homogeneous coordinate on $\mathbb{CP}^1$. The round metric on $S^3$ then reads
\begin{equation}
	g= \mathrm{d}s^2_{\mathbb{CP}^1} + (\sigma + \mathrm{d}\phi)^2
\end{equation}
where 
\begin{equation}
	\begin{split}
		\mathrm{d}s^2_{\mathbb{CP}^1} &= \frac1{(1+|\xi|^2)^2}\mathrm{d}\xi\,\mathrm{ d}\ov{\xi}\\
		\sigma&=\frac{i}{2(1+|\xi|^2)}(\xi \mathrm{d} \ov{\xi}-\ov{\xi}\mathrm{d}\xi)\\ 
		vol_{S^3}&=vol_{\mathbb{CP}^1}\wedge \mathrm{d}\psi .
	\end{split}
\end{equation}
The deformation does not affect $S_+^3$, but it mixes the $S^1$ factor in the internal space with the Hopf fibre of $S_-^3$. It is parametrised by a real number $\Omega$, such that the metric reads
\begin{equation}
	\begin{split}
		\mathrm{d}s^2 &= \alpha^2(\mathrm{d}s^2_{\mathbb{CP}^1}+\kappa^2) + \mathrm{d}s^2_{S_+^3} + e^{-4\Omega}\mathrm{d}\psi^2\\
		\kappa &= \mathrm{d}\phi + \sigma - \frac1{\alpha}e^{-2\Omega}\sqrt{e^{2\Omega}-1}\d\psi.
	\end{split}
\end{equation}
 The three form flux gets changed according to the prescription $\d\phi \rightarrow \d\phi -  \frac1{\alpha}e^{-2\Omega}\sqrt{e^{2\Omega}-1}\d \psi$ and we find:
\begin{equation}
	H_{(3)}=\alpha^2vol_{\mathbb{CP}^1}\wedge (\d\phi - \frac1{\alpha}e^{-2\Omega}\sqrt{e^{2\Omega}-1}\d\psi) + vol_{S_+^3}.
\end{equation}

%
%In spherical coordinates:
%\begin{equation}
%	\begin{aligned}
%		\d s^{2} &= \d s^2\left({\rm AdS}_{3}\right)+\d s^2\left(S^{3}_{+}\right)+\alpha^{2}\,\d s^2\left(S^{3}_{-}\right) +e^{-2\omega}\,\d\psi^{2}\\
%		&\quad+2\,\alpha\,e^{-2\omega}\,\sqrt{-1+e^{2\omega}}\left(-\cos^{2}(\theta)\,\d\varphi_{1}+\sin^{2}(\theta)\,\d\varphi_{2}\right)\,\d\psi\\[5pt]
%		&= \d s^2\left({\rm AdS}_{3}\right)+\d s^2\left(S^{3}_{+}\right)+e^{-4\omega}\,\d\psi^{2}\\
%		&\quad+\alpha^{2}\bigg[\d \theta^2+\cos^{2}(\theta)\left(\d\varphi_{1}-\frac{e^{-2\omega}}{\alpha}\sqrt{-1+e^{2\omega}}\,\d\psi\right)^{2}+\sin^{2}(\theta)\left(\d\varphi_{2}+\frac{e^{-2\omega}}{\alpha}\sqrt{-1+e^{2\omega}}\,\d\psi\right)^{2}\bigg] ,\\[5pt]
%		\d s^2\left(S^{3}_{-}\right) &= \d\theta^{2}+\cos^{2}(\theta)\,\d\varphi_{1}^{2}+\sin^{2}(\theta)\,\d\varphi_{2}^{2}.
%	\end{aligned}
%\end{equation}

\paragraph{Non-SUSY deformation}
Here we study a more general deformation depending on three parameters $\Omega$, $\zeta_{1,8}$ and $\zeta_{2,3}$. It is convenient to write down the uplift in terms of spherical coordinates on $S_-^3$, while again $S_+^3$ is not deformed.\\
\noindent For the metric we find
%\begin{equation}
%	\begin{aligned}
%		\d s^{2} &= \d s^2\left({\rm AdS}_{3}\right)+\d s^2\left(S^{3}_{+}\right) \\
%		&\quad + \alpha^{2}\left(\d\theta^{2}+\Delta\left(2\,e^{2\Omega}+\zeta_{2,3}^{2}\right)\,\cos^{2}(\theta)\,\d\varphi^{2}_{1}+\Delta\left(2+e^{2\Omega}\,\zeta_{1,8}^{2}\right)\,\sin^{2}(\theta)\,\d\varphi_{2}^{2}\right)\\
%		&\quad+2\sqrt{2}\,\alpha\,\Delta\left(-\zeta_{2,3}\,\cos^{2}(\theta)\,\d\varphi_{1}+\zeta_{1,8}\,\sin^{2}(\theta)\,\d\varphi_{2}\right)\,\d\psi \\
%		&\quad+2\,\Delta\left(\cos^{2}(\theta)+e^{2\Omega}\,\sin^{2}(\theta)\right)\,\d\psi^{2}.\\[5pt]
%		\Delta^{-1} &= 2\,e^{2\Omega}+\zeta_{2,3}^{2}+\cos^{2}(\theta)\left(2-2\,e^{2\Omega}-\zeta_{2,3}^{2}+e^{2\Omega}\,\zeta_{1,8}^{2}\right).
%	\end{aligned}
%\end{equation}
\begin{equation}
	\begin{aligned}
		\d s^{2} &= \d s^2\left({\rm AdS}_{3}\right)+\d s^2\left(S^{3}_{+}\right) \\
		&\quad + \alpha^{2}\left(\d\theta^{2}+\Delta\left(2\,e^{2\Omega}+\zeta_{2,3}^{2}\right)\,\cos^{2}(\theta)\,\d\varphi^{2}_{1}+\Delta\left(2+e^{2\Omega}\,\zeta_{1,8}^{2}\right)\,\sin^{2}(\theta)\,\d\varphi_{2}^{2}\right)\\
		&\quad+2\sqrt{2}\,\alpha\,\Delta\left(-\zeta_{2,3}\,\cos^{2}(\theta)\,\d\varphi_{1}+\zeta_{1,8}\,\sin^{2}(\theta)\,\d\varphi_{2}\right)\,\d\psi \\
		&\quad+2\,\Delta\left(\cos^{2}(\theta)+e^{2\Omega}\,\sin^{2}(\theta)\right)\,\d\psi^{2},
	\end{aligned}
\end{equation}
where $\Delta$ is the warp factor
\begin{equation}
		\Delta^{-1} = 2\,e^{2\Omega}+\zeta_{2,3}^{2}+\cos^{2}(\theta)\left(2-2\,e^{2\Omega}-\zeta_{2,3}^{2}+e^{2\Omega}\,\zeta_{1,8}^{2}\right).
\end{equation}
The three-form flux is 
\begin{equation}
	H_{(3)}=vol_{S_-^3}+\alpha ^2\sin(\theta)\cos (\theta)(2+e^{2\Omega}\zeta^2_{1,8})(2e^{2\Omega}+\zeta_{2,3}^2)\Delta^2\mathrm{d}\theta\wedge\kappa_1\wedge\kappa_2,
\end{equation}
with the two one forms $\kappa_{1,2}$ defined by 
\begin{equation}
\begin{split}
		\kappa_1&=\mathrm{d}\phi_1-\frac{\zeta_{2,3}}{\alpha(2e^{2\Omega}+\zeta_{2,3}^2)}\mathrm{d}\psi \\
	\kappa_2&=\mathrm{d}\phi_2+ \frac{\zeta_{1,8}}{\alpha(2+e^{2\Omega}\zeta_{1,8}^2)}\mathrm{d}\psi.
\end{split}
\end{equation}
Note that setting $...$ recovers the supersymmetric deformation described earlier.
%In Hopf coordinates:
% \begin{equation}
% \begin{split}
% ds^2 &= \alpha^2 \left[ d\alpha_1^2 +  ds^2_{\mathbb{CP}^1} + \kappa^2 \right] + ds^2_{\tilde{S}^3} + e^{-4\omega} d\psi^2 \,, \\
% \kappa &= d\theta + \sigma - \frac1{\alpha} e^{-2\omega} \sqrt{1-e^{2\omega}} d\psi \,, \\
% d\sigma &= vol_{\mathbb{CP}^1} \,.
% \end{split}
% \end{equation}
% \begin{equation}
% 	\begin{aligned}
% 		\d s^{2} &= \d s^2\left({\rm AdS}_{3}\right)+\d s^2\left(S^{3}_{+}\right) \\
% 		&\quad + \Delta\,\alpha^{2}\bigg[\frac{1}{\left(1+\vert z\vert^{2}\right)^{2}}\Big(\left(2+e^{2\Omega}\zeta_{1,8}^{2}\right)\left(1+(2+\vert z\vert^{2})\,z_{\rm I}^{2}\right)+\left(2\,e^{2\Omega}+\zeta_{2,3}^{2}\right)\,z_{\rm R}^{2}\Big)\,\d z_{\rm R}^{2}\\
% 		&\qquad\ \  + \frac{1}{\left(1+\vert z\vert^{2}\right)^{2}}\Big(\left(2+e^{2\Omega}\zeta_{1,8}^{2}\right)\left(1+(2+\vert z\vert^{2})\,z_{\rm R}^{2}\right)+\left(2\,e^{2\Omega}+\zeta_{2,3}^{2}\right)\,z_{\rm I}^{2}\Big)\,\d z_{\rm I}^{2}\\
% 		&\qquad\ \  + 2\,\frac{1}{\left(1+\vert z\vert^{2}\right)^{2}}\Big(-\left(2+e^{2\Omega}\zeta_{1,8}^{2}\right)\left(2+\vert z\vert^{2}\right)+2\,e^{2\Omega}+\zeta_{2,3}^{2}\Big)\,\d z_{\rm R}\d z_{\rm I}\\
% 		&\qquad\ \  +2\left(2+e^{2\Omega}\zeta_{1,8}^{2}\right)\left(z_{\rm I}\,\d z_{\rm R}-z_{\rm R}\,\d z_{\rm I}\right)\d\theta+\Big(\left(2+e^{2\Omega}\zeta_{1,8}^{2}\right)\,\vert z\vert^{2}+2\,e^{2\Omega}+\zeta_{2,3}^{2}\Big)\,\d\theta^{2}\bigg]\\
% 		&\quad + \Delta\,\alpha\bigg[-2\sqrt{2}\,e^{2\Omega}\,\zeta_{1,8}\,\left(z_{\rm I}\,\d z_{\rm R}-z_{\rm R}\,\d z_{\rm I}\right)\,\d\psi-2\sqrt{2}\left(e^{2\Omega}\,\zeta_{1,8}\,\vert z\vert^{2}+\zeta_{2,3}\right)\,\d\theta\,\d\psi\bigg]\\
% 		&\quad + 2\,\Delta\left(1+e^{2\Omega}\,\vert z\vert^{2}\right)\,\d\psi^{2} ,\\[5pt]
% 		\Delta^{-1} &= 2+e^{2\Omega}\zeta_{1,8}^{2}+\left(2\,e^{2\Omega}+\zeta_{2,3}^{2}\right)\,\vert z\vert^{2}.
% 	\end{aligned}
% \end{equation}

%%%%

%\begin{equation}
%	\begin{aligned}
%		\d s^{2} &= \d s^2\left({\rm AdS}_{3}\right)+\d s^2\left(S^{3}_{+}\right) \\
%		&\quad + \frac{\alpha^{2}}{1+\gamma\,\vert z\vert^{2}}\bigg[\Big(\left(1+\vert z\vert^{2}\right)\left(1+z_{\rm I}^{2}\right)+\left(\gamma-1\right)\,z_{\rm R}^{2}\Big)\,\frac{\d z_{\rm R}^{2}}{\left(1+\vert z\vert^{2}\right)^{2}}\\
%		&\qquad + \Big(\left(1+\vert z\vert^{2}\right)\left(1+z_{\rm R}^{2}\right)+\left(\gamma-1\right)\,z_{\rm I}^{2}\Big)\,\frac{\d z_{\rm I}^{2}}{\left(1+\vert z\vert^{2}\right)^{2}} - 2\,\Big(1+\vert z\vert^{2}-\left(\gamma-1\right)\Big)\,\frac{\d z_{\rm R}\d z_{\rm I}}{\left(1+\vert z\vert^{2}\right)^{2}}\\
%		&\qquad+2\left(z_{\rm I}\,\d z_{\rm R}-z_{\rm R}\,\d z_{\rm I}\right)\d\theta+\left(\gamma+\vert z\vert^{2}\right)\d\theta^{2}\bigg]\\
%		&\quad + \Delta\,\alpha\bigg[-2\sqrt{2}\,e^{2\Omega}\,\zeta_{1,8}\,\left(z_{\rm I}\,\d z_{\rm R}-z_{\rm R}\,\d z_{\rm I}\right)\,\d\psi-2\sqrt{2}\left(e^{2\Omega}\,\zeta_{1,8}\,\vert z\vert^{2}+\zeta_{2,3}\right)\,\d\theta\,\d\psi\bigg]\\
%		&\quad + 2\,\Delta\left(1+e^{2\Omega}\,\vert z\vert^{2}\right)\,\d\psi^{2} ,\\[5pt]
%		\gamma &= \frac{2\,e^{2\Omega}+\zeta_{2,3}^{2}}{2+e^{2\Omega}\zeta_{1,8}^{2}},\qquad \Delta^{-1} = 2+e^{2\Omega}\zeta_{1,8}^{2}+\left(2\,e^{2\Omega}+\zeta_{2,3}^{2}\right)\,\vert z\vert^{2}.
%	\end{aligned}
%\end{equation}


\subsection{Compactification of the moduli space}

\section{Conclusions} \label{s:Conclusions}

\section*{Acknowledgements}
We are grateful to XXX for useful discussions and correspondence. CE is supported by the FWO-Vlaanderen through the project G006119N and by the Vrije Universiteit Brussel through the Strategic Research Program ``High-Energy Physics''. MG and EM are supported by the Deutsche Forschungsgemeinschaft (DFG, German Research Foundation) via the Emmy Noether program ``Exploring the landscape of string theory flux vacua using exceptional field theory'' (project number 426510644).
	
\bibliographystyle{JHEP}
\bibliography{NewBib}
	
\end{document}